\documentclass{article}
\usepackage{amsmath}
\usepackage{bm}
\usepackage[top=2.5cm, bottom=2.5cm, left=2.75cm, right=2.75cm]{geometry}
\usepackage[utf8x]{inputenc}
\usepackage[slovak,english]{babel}
\usepackage[dvips]{graphicx}
\usepackage{wrapfig}
\usepackage[usenames,dvipsnames]{color}
\usepackage{epstopdf}
\usepackage[font=small,margin=0.5cm]{caption}
\usepackage{float}
\usepackage{csquotes}
\usepackage[
	backend=biber]{biblatex}

\begin{document}
Let us consider the conservation law
\begin{equation}
	\label{conservation law}
	\frac{\partial}{\partial t} \phi(\boldsymbol{x},t)
	+ \nabla \cdot [\phi(\boldsymbol{x},t) \boldsymbol{u}(\boldsymbol{x})] = 0.
\end{equation}
We integrate in space and time
\[
	\int_{t^{n-1}}^{t^n} \int_{\Omega_p}
	\frac{\partial}{\partial t} \phi(\boldsymbol{x},t)
	+\int_{t^{n-1}}^{t^n} \int_{\Omega_p}
	\nabla \cdot [\phi(\boldsymbol{x},t) \boldsymbol{u}(\boldsymbol{x})] = 0.
\]
In the first term we change the order of integration
and apply the Newton-Leibniz theorem for the time integration
\[
	\int_{t^{n-1}}^{t^n} \int_{\Omega_p}
	\frac{\partial}{\partial t} \phi(\boldsymbol{x},t) =
	\int_{\Omega_p} \int_{t^{n-1}}^{t^n}
	\frac{\partial}{\partial t} \phi(\boldsymbol{x},t) =
	\int_{\Omega_p}
	\left(\phi(\boldsymbol{x},t^n) - \phi(\boldsymbol{x},t^{n-1})\right)
	\approx |\Omega_p|(\phi_p^{n} - \phi_p^{n-1}),
\]
where we approximate the cell average at time $ t^n $
by the cell center values
$ \int_{\Omega_p} \phi(\boldsymbol{x},t^{n})
\approx |\Omega_p|\phi^{n}_p. $ \\
In the second term of \eqref{conservation law} we apply the divergence theorem for the volume integral and discretize in space
\[
	\int_{\Omega_p}
	\nabla \cdot [\phi(\boldsymbol{x},t) \boldsymbol{u}(\boldsymbol{x})] =
	\int_{\partial\Omega_p}
	\phi(\boldsymbol{x},t) \boldsymbol{u}(\boldsymbol{x}) \cdot \boldsymbol{n}(\boldsymbol{x}) =
	\sum_{f \in \mathcal{F}_p} \int_{e_f}
	\phi(\boldsymbol{x},t) \boldsymbol{u}(\boldsymbol{x}) \cdot \boldsymbol{n}_{pf}
	\approx
	\sum_{f \in \mathcal{F}_p}
	\phi(\boldsymbol{x}_{pf},t) \int_{e_f} \boldsymbol{u}(\boldsymbol{x}) \cdot \boldsymbol{n}_{pf}
\]
where $ \boldsymbol{n}_{pf} $ is an outward facing normal vector, $ \boldsymbol{x}_{pf} $ is the center of the face and we denote
\[
	a_{pf} \equiv \boldsymbol{u}(\boldsymbol{x}_{pf}) \cdot \boldsymbol{n}_{pf}
	\approx
	\int_{e_f} \boldsymbol{u} \cdot \boldsymbol{n}_{pf}.
\]
Next we integrate this term in time
\[
	\int_{t^{n-1}}^{t^n} \sum_{f \in \mathcal{F}_p} \phi(\boldsymbol{x}_{pf},t) a_{pf} =
	\sum_{f \in \mathcal{F}_p} \int_{t^{n-1}}^{t^n} \phi(\boldsymbol{x}_{pf},t) a_{pf}
	\approx
	\Delta t \sum_{f \in \mathcal{F}_p} \phi(\boldsymbol{x}_{pf},t^{n-1/2}) a_{pf},
\]
where we approximate the time integral of $ \phi $ using the midpoint rule. \\
The discretization of \eqref{conservation law} thus reads as
\begin{equation}
	\frac{|\Omega_p|}{\Delta t}(\phi_p^n - \phi_p^{n-1})
	+ \sum_{f \in \mathcal{F}_p} \phi(\boldsymbol{x}_{pf},t^{n-1/2}) a_{pf} = 0.
\end{equation}
Next we approximate the value at the face center at half-time step
\[
	\phi(\boldsymbol{x}_{pf},t^{n-1/2})
	\approx
	\frac{1}{2}(\phi(\boldsymbol{x}_{pf},t^n) + \phi(\boldsymbol{x}_{pf},t^{n-1}))
	\equiv
	\phi_{f}^{n-1/2}.
\]
For the outflow case $ a_{pf} \geq 0 $
\[
	\phi_{f}^{n-1/2} = \frac{1}{2}(\phi_{pf}^n + \phi_{qf}^{n-1}),
\]
where
\[
	\phi_{pf}^{n} = \phi_p^{n} + \nabla\phi_p^{n}\cdot\boldsymbol{d}_{pf},
	\quad \text{and} \quad
	\phi_{qf}^{n-1} = \phi_q^{n-1} + \nabla\phi_q^{n-1}\cdot\boldsymbol{d}_{qf}.
\]
We take the average of reconstructed face values from neighboring cells,
the upwind value is treated implicitly, and the downwind explicitly.
This mimics the approximation of the face value of the IIOE method in 1D:
\[
	\phi_{i+1/2}^{n-1/2} = \frac{1}{2}(\phi_i^{n} + \phi_{i+1}^{n-1}).
\]
For the outflow case $ a_{pf} < 0 $
\[
	\phi_{f}^{n-1/2} = \frac{1}{2}(\phi_{pf}^{n-1} + \phi_{qf}^{n}),
\]
where
\[
	\phi_{pf}^{n-1} = \phi_p^{n-1} + \nabla\phi_p^{n-1}\cdot\boldsymbol{d}_{pf},
	\quad \text{and} \quad
	\phi_{qf}^{n} = \phi_q^{n} + \nabla\phi_q^{n}\cdot\boldsymbol{d}_{qf}.
\]
We divide the set of faces to inflow and outflow parts
\begin{align*}
	\mathcal{B}_p^- = \left\{b \in \mathcal{B}_p\, |\, a_{pb} < 0\right\}\
	\quad &\text{and} \quad
	\mathcal{B}_p^+ = \mathcal{B}_p \setminus \mathcal{B}_p^- \\
	\mathcal{F}_p^- = \left\{f \in (\mathcal{F}_p \setminus \mathcal{B}_p) \, |\, a_{pf} < 0\right\}
	\quad &\text{and} \quad
	\mathcal{F}_p^+ = (\mathcal{F}_p \setminus \mathcal{B}_p) \setminus \mathcal{F}_p^-
\end{align*}
The discretization of \eqref{conservation law} takes the form
\begin{align}
	\nonumber
	\frac{|\Omega_p|}{\Delta t}(\phi_p^{n} - \phi_p^{n-1})
	&+ \sum_{f \in \mathcal{F}_p^-} \phi_{pf}^{n-1/2} a_{pf}
	+ \sum_{f \in \mathcal{F}_p^+} \phi_{pf}^{n-1/2} a_{pf}\\
	&+ \sum_{f \in \mathcal{B}_p^-} \phi_{pb}^{n-1/2} a_{pb}
	+ \sum_{f \in \mathcal{B}_p^+} \phi_{pb}^{n-1/2} a_{pb} = 0,
\end{align}
and the face values depend on the sign of flux $ a_{pf} $
\[
	\phi_{pf}^{n-1/2} =
	\begin{cases}
		\frac{1}{2}(\phi_p^n + \nabla\phi_p^n\cdot\boldsymbol{d}_{pf} +
		\phi_q^{n-1} + \nabla\phi_q^{n-1}\cdot\boldsymbol{d}_{qf}),
		\quad &\text{ if } a_{pf} \geq 0 \\
		\frac{1}{2}(\phi_p^{n-1} + \nabla\phi_p^{n-1}\cdot\boldsymbol{d}_{pf} +
		\phi_q^{n} + \nabla\phi_q^{n}\cdot\boldsymbol{d}_{qf})
		\quad &\text{ if } a_{pf} < 0
	\end{cases}
\]
\[
	\phi_{pb}^{n-1/2} =
	\begin{cases}
		\frac{1}{2}\left(\phi_p^n + \nabla\phi_p^n\cdot\boldsymbol{d}_{pf} +
		\phi_e(\boldsymbol{x}_{pb}, t^{n-1})\right),
		\quad &\text{ if } a_{pf} \geq 0 \\
		\phi_e(\boldsymbol{x}_{pb}, t^{n-1/2}),
		\quad &\text{ if } a_{pf} < 0.
	\end{cases}
\]
Next we describe a strategy to prevent oscillations.\\
\subsection{Outflow case}
For the case $ a_{pf} \geq 0 $ we can rewrite the approximation of the face values as
\begin{equation}
	\label{faceval}
	\begin{split}
		\phi_{pf}^{n-1/2}
		&= \frac{1}{2}(\phi_p^n + \nabla\phi_p^n\cdot\boldsymbol{d}_{pf} +
		\phi_q^{n-1} + \nabla\phi_q^{n-1}\cdot\boldsymbol{d}_{qf}) \\
		&= \phi_p^n + \frac{1}{2}(-\phi_p^n + \nabla\phi_p^n\cdot\boldsymbol{d}_{pf} +
		\phi_q^{n-1} + \nabla\phi_q^{n-1}\cdot\boldsymbol{d}_{qf}),
	\end{split}
\end{equation}
which is now written as the upwind cell center value plus a correction term.
To simplify the notation, we denote the correction term as
\[
	\mathcal{D}^+_p\phi(\boldsymbol{x}_f)
	\equiv
	\frac{1}{2}(-\phi_p^n + \nabla\phi_p^n\cdot\boldsymbol{d}_{pf} +
	\phi_q^{n-1} + \nabla\phi_q^{n-1}\cdot\boldsymbol{d}_{qf}),
\]
the plus sign indicating that $ a_{pf} \geq 0 $ is non-negative.
To ensure boundedness, we apply a limiter $ \Psi_{f} $ to the correction term
\begin{equation} \label{rec_face}
	\phi_{pf}^{n-1/2} = \phi_p^n + \Psi_{f} \mathcal{D}^+_p\phi(\boldsymbol{x}_f)
\end{equation}
%\begin{equation}
%	\begin{split}
%		\phi_{pf}^{n-1/2} &= \phi_p^n + \frac{\Psi_f}{2}(-\phi_p^n + \nabla\phi_p^n\cdot\boldsymbol{d}_{pf} +
%\phi_q^{n-1} + \nabla\phi_q^{n-1}\cdot\boldsymbol{d}_{qf})\\
% &= \left(1 - \frac{\Psi_f}{2}\right)\phi_p^n +
%\frac{\Psi_f}{2}\nabla\phi_p^n\cdot\boldsymbol{d}_{pf} +
%\frac{\Psi_f}{2}(\phi_q^{n-1} + \nabla\phi_q^{n-1}\cdot\boldsymbol{d}_{qf}).
%	\end{split}
%\end{equation}
To compute the limiter $ \Psi_{f} $, we use similar ideas as in [ ].
We want to ensure that the min-max principle is satisfied.
For that purpose, we bound the reconstructed vertex values
between the minimum and maximum values of cell averages in neighboring cells sharing the vertex.\\
Thus, we need to approximate the vertex values. We can do it similarly to face values.
The correction term in \eqref{faceval} is computed at the face coordinate $ \boldsymbol{x}_f $.
It can be similarly calculated at any given point $ \boldsymbol{x} $ by
\[
	\mathcal{D}^+_p\phi (\boldsymbol{x})
	\equiv
	\frac{1}{2}(-\phi_p^n + \nabla\phi_p^n\cdot\boldsymbol{d}_p(\boldsymbol{x}) +
	\phi_q^{n-1} + \nabla\phi_q^{n-1}\cdot\boldsymbol{d}_q(\boldsymbol{x})),
\]
where $ \boldsymbol{d}_p(\boldsymbol{x}) = (\boldsymbol{x} - \boldsymbol{x}_p) $
is the distance of a given point from the cell center.
By substituting the vertex coordinate $ \boldsymbol{x}_v $,
the reconstructed vertex value takes the form
\[
	\phi_{pv}^{n-1/2} =
	\phi_p^n + \Psi_{f}\mathcal{D}^+_p\phi (\boldsymbol{x}_{v}),
\]
which is similar to \eqref{rec_face}, except the point where it is computed.
We want to ensure the min-max principle is satisfied for the vertex values
\[
	\phi_{pv, neighbor}^{min} \leq \phi_{pv}^{n-1/2} \leq \phi_{pv, neighbor}^{max},
\]
where $ \phi_{pv, neighbor}^{min} $ and $ \phi_{pv, neighbor}^{max} $
are the minimum and maximum values (from both time $ t^n $ and $ t^n $)
of the cell averages among all neighboring cells sharing the vertex.
\[
	\phi_{pv, neighbor}^{min}
	\leq
	\phi_p^n + \Psi_{f}\mathcal{D}^+_p\phi (\boldsymbol{x}_{v})
	\leq
	\phi_{pv, neighbor}^{max},
\]
\[
	\phi_{pv, neighbor}^{min} -\phi_p^n
	\leq
	\Psi_{f}\mathcal{D}^+_p\phi (\boldsymbol{x}_{v})
	\leq
	\phi_{pv, neighbor}^{max} -\phi_p^n.
\]
So, if $ \mathcal{D}^+_p\phi (\boldsymbol{x}_{v}) > 0 $ is positive
\[
	\frac{\phi_{v, neighbor}^{min} - \phi_p^n}{\mathcal{D}^+_p\phi (\boldsymbol{x}_{v})}
	\leq
	\Psi_{f}
	\leq
	\frac{\phi_{v, neighbor}^{max} - \phi_p^n}{\mathcal{D}^+_p\phi (\boldsymbol{x}_{v})},
\]
if $ \mathcal{D}^+_p\phi (\boldsymbol{x}_{v}) < 0 $ is negative
\[
	\frac{\phi_{v, neighbor}^{max} - \phi_p^n}{\mathcal{D}^+_p\phi (\boldsymbol{x}_{v})}
	\leq
	\Psi_{f}
	\leq
	\frac{\phi_{v, neighbor}^{min} - \phi_p^n}{\mathcal{D}^+_p\phi (\boldsymbol{x}_{v})}.
\]
If $ \mathcal{D}^+_p\phi (\boldsymbol{x}_{v}) = 0 $ we choose $ \Psi_{f} = 1 $.
Thus, we bound the limiter from above linearlly
\[
	\Psi_{f}
	\leq
	\max \left(
		\frac{\phi_{v, neighbor}^{min} -\phi_p^n}{\mathcal{D}^+_p\phi (\boldsymbol{x}_{v})},
		\frac{\phi_{v, neighbor}^{max} -\phi_p^n}{\mathcal{D}^+_p\phi (\boldsymbol{x}_{v})}
		\right).
\]
We also want the limiter to be bounded by $ 0 $ from below,
in which case the approximation of the face value in \eqref{rec_face}
is the first order implicit upwind scheme, and by $ 1 $ from above,
which is the case of unlimited second order scheme.
We do the computation at each vertex. For a given cell,
we choose the minimum of these calculated $ \Psi $ values
\begin{equation}
	\Psi_{f} =
	\min_{v \in \mathcal{V}_{f}}
	\begin{cases}
		\Phi(r_{v}^+) \quad
		&\text{if} \ \mathcal{D}^+_p\phi (\boldsymbol{x}_{v}) \neq 0\\
		1 \quad &\text{otherwise}.
	\end{cases}
\end{equation}
\[
	r_{v}^+ =
	\max \left(
		\frac{\phi_{v, neighbor}^{min} -\phi_p^n}{\mathcal{D}^+_p\phi (\boldsymbol{x}_{v})},
		 \frac{\phi_{v, neighbor}^{max} -\phi_p^n}{\mathcal{D}^+_p\phi (\boldsymbol{x}_{v})}
		 \right).
\]
We require
\[
	0 \leq \Phi(r_{v}^+) \leq \min(1, r_{v}^+).
\]
The inflow case $ a_{pf} < 0 $ is treated analogously.
We always treat the upwind values implicitly and downwind values explicitly.
\newpage
\subsection{Inflow case}
For the case $ a_{pf} < 0 $ we can rewrite the approximation of the face values as
\begin{equation}
	\begin{split}
		\phi_{pf}^{n-1/2}
		&= \frac{1}{2}(\phi_q^{n} + \nabla\phi_q^{n}\cdot\boldsymbol{d}_{qf} +
		\phi_p^{n-1} + \nabla\phi_p^{n-1}\cdot\boldsymbol{d}_{pf}) \\
		&= \phi_q^{n} + \frac{1}{2}(-\phi_q^{n} + \nabla\phi_q^{n}\cdot\boldsymbol{d}_{qf} +
		\phi_p^{n-1} + \nabla\phi_p^{n-1}\cdot\boldsymbol{d}_{pf}),
	\end{split}
\end{equation}
which is now written, as in the outflow case,
as the upwind cell center value plus a correction term.
To simplify the notation, we denote the correction term as
\[
	\mathcal{D}_{p}^-\phi(\boldsymbol{x}_f)
	\equiv
	\frac{1}{2}(-\phi_q^{n} + \nabla\phi_q^{n}\cdot\boldsymbol{d}_{qf} +
	\phi_p^{n-1} + \nabla\phi_p^{n-1}\cdot\boldsymbol{d}_{pf}),
\]
the minus sign indicating that $ a_{pf} < 0 $ is negative.
To ensure boundedness, we apply a limiter $ \Psi_{f} $ to the correction term
\begin{equation}
	\phi_{pf}^{n-1/2} =
	\phi_q^{n} + \Psi_{f}\mathcal{D}_{p}^-\phi(\boldsymbol{x}_f).
\end{equation}
The reconstructed vertex value takes the form
\[
	\phi_{pv}^{n-1/2} =
	\phi_q^{n} + \Psi_{f}\mathcal{D}_{p}^-\phi(\boldsymbol{x}_{v}).
\]
We compute the limiter in a way to ensure the maximum principle is satisfied for the vertex values
\[
	\phi_{pv, neighbor}^{min}
	\leq
	\phi_{pv}^{n-1/2}
	\leq
	\phi_{pv, neighbor}^{max},
\]
where $ \phi_{pv, neighbor}^{min} $ and $ \phi_{pv, neighbor}^{max} $
are the minimum and maximum values (from both time $ t^n $ and $ t^n $)
of the cell averages among all neighboring cells sharing the vertex.
\[
	\phi_{pv, neighbor}^{min}
	\leq
	\phi_q^{n} + \Psi_{f}\mathcal{D}_{p}^-\phi_{v}
	\leq
	\phi_{pv, neighbor}^{max},
\]
\[
	\phi_{pv, neighbor}^{min} -\phi_q^{n}
	\leq
	\Psi_{f}\mathcal{D}_{p}^-\phi_{v}
	\leq
	\phi_{pv, neighbor}^{max} -\phi_q^{n}.
\]
So, if $ \mathcal{D}_{p}^-\phi_{v} > 0 $ is positive
\[
	\frac{\phi_{v, neighbor}^{min} - \phi_q^{n}}{\mathcal{D}_{p}^-\phi_{v}}
	\leq
	\Psi_{f}
	\leq
	\frac{\phi_{v, neighbor}^{max} - \phi_q^{n}}{\mathcal{D}_{p}^-\phi_{v}},
\]
if $ \mathcal{D}_{p}^-\phi_{v} < 0 $ is negative
\[
	\frac{\phi_{v, neighbor}^{max} - \phi_q^{n}}{\mathcal{D}_{p}^-\phi_{v}}
	\leq
	\Psi_{f}
	\leq
	\frac{\phi_{v, neighbor}^{min} - \phi_q^{n}}{\mathcal{D}_{p}^-\phi_{v}}.
\]
If $ \mathcal{D}_{p}^-\phi_{v} = 0 $ we choose $ \Psi_{f} = 1 $.
Thus, we bound the limiter from above
\[
	\Psi_{f}
	\leq
	\max \left(
		\frac{\phi_{v, neighbor}^{min} -\phi_q^{n}}{\mathcal{D}_{p}^-\phi_{v}},
		\frac{\phi_{v, neighbor}^{max} -\phi_q^{n}}{\mathcal{D}_{p}^-\phi_{v}}
		\right).
\]
We also want the limiter to be bounded by $ 0 $ from below,
in which case the approximation of the face value in \eqref{rec_face}
is the first order implicit upwind scheme, and by $ 1 $ from above,
which is the case of unlimited second order scheme.
We do the computation at each vertex. For a given cell,
we choose the minimum of these calculated $ \Psi $ values
\begin{equation}
	\Psi_{f} =
	\min_{v \in \mathcal{V}_{f}}
	\begin{cases}
		\Phi(r_{v}^-) \quad
		&\text{if} \ \mathcal{D}_{p}^-\phi_{v} \neq 0\\
		1 \quad
		&\text{otherwise}.
	\end{cases}
\end{equation}
\[
	r_{v}^- =
	\max \left(
		\frac{\phi_{v, neighbor}^{min} -\phi_q^{n}}{\mathcal{D}_{p}^-\phi_{v}},
		\frac{\phi_{v, neighbor}^{max} -\phi_q^{n}}{\mathcal{D}_{p}^-\phi_{v}}
		\right).
\]
We require
\[
	0 \leq \Phi(r_{v}^-) \leq \min(1, r_{v}^-).
\]
The equations are solved iteratively using the deferred correction method as in the case of iterative IIOE method.
\subsection{system}
The discretization of \eqref{conservation law} takes the form
\begin{align}
	\nonumber
	\frac{|\Omega_p|}{\Delta t}(\phi_p^{n} - \phi_p^{n-1})
	&+ \sum_{f \in \mathcal{F}_p^-}
	\phi_{pf}^{n-1/2} a_{pf} + \sum_{f \in \mathcal{F}_p^+} \phi_{pf}^{n-1/2} a_{pf}\\
	&+ \sum_{f \in \mathcal{B}_p^-}
	\phi_{pb}^{n-1/2} a_{pb} + \sum_{f \in \mathcal{B}_p^+} \phi_{pb}^{n-1/2} a_{pb} = 0,
\end{align}
and the face values depend on the sign of flux $ a_{pf} $
\[
	\phi_{pf}^{n-1/2} =
	\begin{cases}
		\phi_p^n +
		\frac{\Psi_{f}}{2}(-\phi_p^n + \nabla\phi_p^n\cdot\boldsymbol{d}_{pf} +
		\phi_q^{n-1} + \nabla\phi_q^{n-1}\cdot\boldsymbol{d}_{qf}),\quad
		&\text{ if } a_{pf} \geq 0 \\
		\phi_q^{n} +
		\frac{\Psi_{f}}{2}(-\phi_q^{n} + \nabla\phi_q^{n}\cdot\boldsymbol{d}_{qf} +
		\phi_p^{n-1} + \nabla\phi_p^{n-1}\cdot\boldsymbol{d}_{pf}) \quad
		&\text{ if } a_{pf} < 0
	\end{cases}
\]
% \[ \phi_{pf}^{n-1/2} =
% \begin{cases}
% 	\phi_p^n + \frac{\Psi_{f}}{2}\mathcal{D}^+\phi(\boldsymbol{x}_f), \quad &\text{ if } a_{pf} \geq 0 \\
% 	\phi_q^{n} + \frac{\Psi_{f}}{2}\mathcal{D}^-\phi(\boldsymbol{x}_f) \quad &\text{ if } a_{pf} < 0
% \end{cases} \]
\[
	\phi_{pb}^{n-1/2} =
	\begin{cases}
		\phi_p^n +
		\frac{\Psi_{pb}}{2}\left(-\phi_p^n + \nabla\phi_p^n\cdot\boldsymbol{d}_{pf} +
		\phi_e(\boldsymbol{x}_{pb}, t^{n-1})\right), \quad
		&\text{ if } a_{pf}
		\geq 0 \\
		\phi_e(\boldsymbol{x}_{pb}, t^{n-1/2}),\quad
		&\text{ if } a_{pf} < 0.
	\end{cases}
\]
% \[ \phi_{pb}^{n-1/2} =
% \begin{cases}
% 	\phi_p^n + \frac{\Psi_{pb}}{2}\mathcal{D}^+\phi(\boldsymbol{x}_{pb}), \quad &\text{ if } a_{pf} \geq 0 \\
% 	\phi_e(\boldsymbol{x}_{pb}, t^{n-1/2}),\quad &\text{ if } a_{pf} < 0.
% \end{cases} \]
% \begin{align}
% 	\nonumber
% 	\frac{|\Omega_p|}{\Delta t}(\phi_p^{n} - \phi_p^{n-1})
% 	&+ \sum_{f \in \mathcal{F}_p^-}(\phi_q^{n} + \frac{\Psi_{f}}{2}\mathcal{D}^-\phi(\boldsymbol{x}_f)) a_{pf}
% 	+ \sum_{f \in \mathcal{F}_p^+} (\phi_p^n + \frac{\Psi_{f}}{2}\mathcal{D}^+\phi(\boldsymbol{x}_f)) a_{pf}\nonumber\\
% 	&+ \sum_{f \in \mathcal{B}_p^-} \phi_e(\boldsymbol{x}_{pb}, t^{n-1/2}) a_{pb}
% 	+ \sum_{f \in \mathcal{B}_p^+} (\phi_p^n + \frac{\Psi_{pb}}{2}\mathcal{D}^+\phi(\boldsymbol{x}_{pb})) a_{pb} = 0,
% \end{align}
% \begin{align}
% 	\nonumber
% 	\frac{|\Omega_p|}{\Delta t}(\phi_p^{n} - \phi_p^{n-1})
% 	&+ \sum_{f \in \mathcal{F}_p^-}(\phi_q^{n} + \frac{\Psi_{f}}{2}(-\phi_q^{n} + \nabla\phi_q^{n}\cdot\boldsymbol{d}_{qf})) a_{pf}
% 	+ \sum_{f \in \mathcal{F}_p^+} (\phi_p^n + \frac{\Psi_{f}}{2}(-\phi_p^n + \nabla\phi_p^n\cdot\boldsymbol{d}_{pf})) a_{pf}\nonumber\\
% 	&+ \sum_{f \in \mathcal{B}_p^+} (\phi_p^n + \frac{\Psi_{pb}}{2}(-\phi_p^n + \nabla\phi_p^n\cdot\boldsymbol{d}_{pf})) a_{pb} = \nonumber\\
% 	&-\sum_{f \in \mathcal{F}_p^-}\frac{\Psi_{f}}{2}(\phi_p^{n-1} + \nabla\phi_p^{n-1}\cdot\boldsymbol{d}_{pf}) a_{pf}
% 	-\sum_{f \in \mathcal{F}_p^+} \frac{\Psi_{f}}{2}(\phi_q^{n-1} + \nabla\phi_q^{n-1}\cdot\boldsymbol{d}_{qf}) a_{pf}\nonumber\\
% 	&- \sum_{f \in \mathcal{B}_p^-} \phi_e(\boldsymbol{x}_{pb}, t^{n-1/2}) a_{pb}
% 	-\sum_{f \in \mathcal{B}_p^+}\frac{\Psi_{pb}}{2}\phi_e(\boldsymbol{x}_{pb}, t^{n-1}) a_{pb},
% 	\nonumber
% \end{align}
% \begin{align}
% 	\nonumber
% 	(\phi_p^{n} - \phi_p^{n-1})
% 	&+ \sum_{f \in \mathcal{F}_p^-}(\phi_q^{n} + \frac{\Psi_{f}}{2}(-\phi_q^{n} + \nabla\phi_q^{n}\cdot\boldsymbol{d}_{qf})) \frac{a_{pf}\Delta t}{|\Omega_p|}
% 	+ \sum_{f \in \mathcal{F}_p^+} (\phi_p^n + \frac{\Psi_{f}}{2}(-\phi_p^n + \nabla\phi_p^n\cdot\boldsymbol{d}_{pf})) \frac{a_{pf}\Delta t}{|\Omega_p|}\nonumber\\
% 	&+ \sum_{f \in \mathcal{B}_p^+} (\phi_p^n + \frac{\Psi_{pb}}{2}(-\phi_p^n + \nabla\phi_p^n\cdot\boldsymbol{d}_{pf})) \frac{a_{pb}\Delta t}{|\Omega_p|} = \nonumber\\
% 	&-\sum_{f \in \mathcal{F}_p^-}\frac{\Psi_{f}}{2}(\phi_p^{n-1} + \nabla\phi_p^{n-1}\cdot\boldsymbol{d}_{pf}) \frac{a_{pf}\Delta t}{|\Omega_p|}
% 	-\sum_{f \in \mathcal{F}_p^+} \frac{\Psi_{f}}{2}(\phi_q^{n-1} + \nabla\phi_q^{n-1}\cdot\boldsymbol{d}_{qf}) \frac{a_{pf}\Delta t}{|\Omega_p|}\nonumber\\
% 	&- \sum_{f \in \mathcal{B}_p^-} \phi_e(\boldsymbol{x}_{pb}, t^{n-1/2}) \frac{a_{pb}\Delta t}{|\Omega_p|}
% 	-\sum_{f \in \mathcal{B}_p^+}\frac{\Psi_{pb}}{2}\phi_e(\boldsymbol{x}_{pb}, t^{n-1}) \frac{a_{pb}\Delta t}{|\Omega_p|},
% 	\nonumber
% \end{align}
\begin{align}
	\nonumber
	\frac{|\Omega_p|}{\Delta t}(\phi^{n,k}_p - \phi_p^{n-1})
	&+ \sum_{f \in \mathcal{F}_p^-}
	(\phi_q^{n,k} +
	\frac{\Psi_{f}^{k-1}}{2}(-\phi_q^{n,k-1} + \nabla\phi_q^{n,k-1}\cdot\boldsymbol{d}_{qf} +
	\phi_p^{n-1} + \nabla\phi_p^{n-1}\cdot\boldsymbol{d}_{pf})) a_{pf}
	\nonumber\\
	&+ \sum_{f \in \mathcal{F}_p^+}
	(\phi_p^{n,k} +
	\frac{\Psi_{f}^{k-1}}{2}(-\phi_p^{n,k-1} + \nabla\phi_p^{n,k-1}\cdot\boldsymbol{d}_{pf} +
	\phi_q^{n-1} + \nabla\phi_q^{n-1}\cdot\boldsymbol{d}_{qf})) a_{pf}
	\nonumber\\
	&+ \sum_{f \in \mathcal{B}_p^-}
	\phi_e(\boldsymbol{x}_{pb}, t^{n-1/2}) a_{pb}
	\nonumber\\
	&+ \sum_{f \in \mathcal{B}_p^+}
	\phi_p^{n,k} +
	\frac{\Psi_{pb}^{k-1}}{2}\left(-\phi_p^{n,k-1} + \nabla\phi_p^{n,k-1}\cdot\boldsymbol{d}_{pf} +
	\phi_e(\boldsymbol{x}_{pb}, t^{n-1})\right) a_{pb} = 0,
\end{align}
% \begin{align}
% 	\nonumber
% 	\frac{|\Omega_p|}{\Delta t}(\phi^{n,k}_p - \phi_p^{n-1})
% 	&+ \sum_{f \in \mathcal{F}_p^-}(\phi_q^{n,k} + \nabla\phi_q^{n,k-1}\cdot\boldsymbol{d}_{qf} +
% 	\phi_p^{n-1} + \nabla\phi_p^{n-1}\cdot\boldsymbol{d}_{pf}) \frac{a_{pf}}{2}\nonumber\\
% 	&+ \sum_{f \in \mathcal{F}_p^+} (\phi_p^{n,k} + \nabla\phi_p^{n,k-1}\cdot\boldsymbol{d}_{pf} +
% 	\phi_q^{n-1} + \nabla\phi_q^{n-1}\cdot\boldsymbol{d}_{qf}) \frac{a_{pf}}{2}\nonumber\\
% 	&+ \sum_{f \in \mathcal{B}_p^-} \phi_e(\boldsymbol{x}_{pb}, t^{n-1/2}) a_{pb}\nonumber \\
% 	&+ \sum_{f \in \mathcal{B}_p^+}\left(\phi_p^{n,k} + \nabla\phi_p^{n,k-1}\cdot\boldsymbol{d}_{pf} +
% 	\phi_e(\boldsymbol{x}_{pb}), t^{n-1}\right) \frac{a_{pb}}{2} = 0,
% \end{align}
% \begin{align}
% 	\nonumber
% 	\frac{|\Omega_p|}{\Delta t}\phi^{n,k}_p
% 	+ \sum_{f \in \mathcal{F}_p^-}(\phi_q^{n,k} + \frac{\Psi_{f}^{k-1}}{2}(-\phi_q^{n,k})) a_{pf}
% 	+ \sum_{f \in \mathcal{F}_p^+} (\phi_p^{n,k} + \frac{\Psi_{f}^{k-1}}{2}(-\phi_p^{n,k})) a_{pf}
% 	+ \sum_{f \in \mathcal{B}_p^+} (\phi_p^{n,k} + \frac{\Psi_{pb}^{k-1}}{2}\left(-\phi_p^{n,k}\right)) a_{pb} =\nonumber\\
% 	\frac{|\Omega_p|}{\Delta t}\phi_p^{n-1} - \sum_{f \in \mathcal{F}_p^-} \frac{\Psi_{f}^{k-1}}{2}(\nabla\phi_q^{n,k-1}\cdot\boldsymbol{d}_{qf} +
% 	\phi_p^{n-1} + \nabla\phi_p^{n-1}\cdot\boldsymbol{d}_{pf}) a_{pf} -
% 	\sum_{f \in \mathcal{F}_p^+} \frac{\Psi_{f}^{k-1}}{2}( \nabla\phi_p^{n,k-1}\cdot\boldsymbol{d}_{pf} +
% 	\phi_q^{n-1} + \nabla\phi_q^{n-1}\cdot\boldsymbol{d}_{qf}) a_{pf} \nonumber\\
% 	-\sum_{f \in \mathcal{B}_p^+}\frac{\Psi_{pb}^{k-1}}{2}\left(\nabla\phi_p^{n,k-1}\cdot\boldsymbol{d}_{pf} +
% 	\phi_e(\boldsymbol{x}_{pb}, t^{n-1})\right) a_{pb} -
% 	\sum_{f \in \mathcal{B}_p^-} \phi_e(\boldsymbol{x}_{pb}, t^{n-1/2}) a_{pb},
% \end{align}
\begin{equation}
	\Psi_{f}^{k-1} = \min_{v \in \mathcal{V}_p}
	\begin{cases}
		\Phi(r_{v}^{+,k-1}) \quad
		&\text{if} \ \mathcal{D}^+_{p}\phi^{*}(\boldsymbol{x}_v)
		\neq 0\\
		1 \quad &\text{otherwise}.
	\end{cases}
\end{equation}
\[
	r_{v}^{+,k-1} =
	\max \left(
		\frac{\phi_{v, neighbor}^{min,k-1} -\phi_p^{n,k-1}}{\mathcal{D}^+_{p}\phi^{k-1}(\boldsymbol{x}_v)},
		\frac{\phi_{v, neighbor}^{max,k-1} -\phi_p^{n,k-1}}{\mathcal{D}^+_{p}\phi^{k-1}(\boldsymbol{x}_v)}
		\right).
\]
We require
\[
	0 \leq \Phi(r_{v}^{+,k-1}) \leq \min(1, r_{v}^{+,k-1}).
\]
\[
	\mathcal{D}^+_{p}\phi^{k-1}(\boldsymbol{x}_v) =
	-\phi_p^{n,k-1} + \nabla\phi_p^{n,k-1}\cdot\boldsymbol{d}_{pf} +
	\phi_q^{n-1} + \nabla\phi_q^{n-1}\cdot\boldsymbol{d}_{qf}
\]
Similarly
\begin{equation}
	\Psi_{f}^{k-1} = \min_{v \in \mathcal{V}_p}
	\begin{cases}
		\Phi(r_{v}^{-,k-1}) \quad
		&\text{if} \quad \mathcal{D}^-_p\phi^{k-1}(\boldsymbol{x}_v)
		\neq 0\\
		1 \quad &\text{otherwise}.
	\end{cases}
\end{equation}
\[
	r_{v}^{-,k-1} =
	\max \left(
	\frac{\phi_{v, neighbor}^{min,k-1} -\phi_p^{n,k-1}}{\mathcal{D}^-_p\phi^{k-1}(\boldsymbol{x}_v)},
	\frac{\phi_{v, neighbor}^{max,k-1} -\phi_p^{n,k-1}}{\mathcal{D}^-_p\phi^{k-1}(\boldsymbol{x}_v)}
	\right).
\]
We require
\[
	0
	\leq
	\Phi(r_{v}^{-,k-1})
	\leq
	\min(1, r_{v}^{-,k-1}).
\]
\[
	\mathcal{D}^-_p\phi^{k-1}(\boldsymbol{x}_v) =
	-\phi_q^{n,k-1} + \nabla\phi_q^{n,k-1}\cdot\boldsymbol{d}_{qf} +
	\phi_p^{n-1} + \nabla\phi_p^{n-1}\cdot\boldsymbol{d}_{pf},
\]
where $ k = 1,\dots, K $, $ \phi_p^{n, 0} = \phi_p^{n-1}. $
\end{document}