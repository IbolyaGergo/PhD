\documentclass[../thesis.tex]{subfiles}
\begin{document}
Let us consider the conservation law
\begin{equation}\label{eqn:scalar-conservation-law-3d}
	\frac{\partial}{\partial t} \phi(\boldsymbol{x},t)
	+ \nabla \cdot \left( \phi(\boldsymbol{x},t) \boldsymbol{u}(\boldsymbol{x}) \right) = 0.
\end{equation}
We integrate in space and time
\[
	\int_{t^{n-1}}^{t^n} \int_{\Omega_p}
	\frac{\partial}{\partial t} \phi(\boldsymbol{x},t)
	+\int_{t^{n-1}}^{t^n} \int_{\Omega_p}
	\nabla \cdot \left( \phi(\boldsymbol{x},t) \boldsymbol{u}(\boldsymbol{x}) \right) = 0.
\]
In the first term we change the order of integration
and apply the Newton-Leibniz theorem for the time integration
\[
	\int_{t^{n-1}}^{t^n} \int_{\Omega_p}
	\frac{\partial}{\partial t} \phi(\boldsymbol{x},t) =
	\int_{\Omega_p} \int_{t^{n-1}}^{t^n}
	\frac{\partial}{\partial t} \phi(\boldsymbol{x},t) =
	\int_{\Omega_p}
	\left(\phi(\boldsymbol{x},t^n) - \phi(\boldsymbol{x},t^{n-1})\right)
	\approx |\Omega_p|(\phi_p^{n} - \phi_p^{n-1}),
\]
where we approximate the cell average at time $ t^n $
by the cell center values
\(\int_{\Omega_p} \phi(\boldsymbol{x},t^{n})~\approx~|\Omega_p|\phi^{n}_p.\) \\
In the second term of \eqref{eqn:scalar-conservation-law-3d} we apply the divergence theorem for the volume integral and discretize in space
\[
	\int_{\Omega_p}
	\nabla \cdot [\phi(\boldsymbol{x},t) \boldsymbol{u}(\boldsymbol{x})] =
	\int_{\partial\Omega_p}
	\phi(\boldsymbol{x},t) \boldsymbol{u}(\boldsymbol{x}) \cdot \boldsymbol{n}(\boldsymbol{x}) =
	\sum_{f \in \mathcal{F}_p} \int_{e_f}
	\phi(\boldsymbol{x},t) \boldsymbol{u}(\boldsymbol{x}) \cdot \boldsymbol{n}_{pf}
	\approx
	\sum_{f \in \mathcal{F}_p}
	\phi(\boldsymbol{x}_{pf},t) \int_{e_f} \boldsymbol{u}(\boldsymbol{x}) \cdot \boldsymbol{n}_{pf}
\]
where $ \boldsymbol{n}_{pf} $ is an outward facing normal vector, $ \boldsymbol{x}_{pf} $ is the center of the face, and we denote
\[
	a_{pf} \equiv \boldsymbol{u}(\boldsymbol{x}_{pf}) \cdot \boldsymbol{n}_{pf}
	\approx
	\int_{e_f} \boldsymbol{u} \cdot \boldsymbol{n}_{pf}.
\]
Next we integrate this term in time
\[
	\int_{t^{n-1}}^{t^n} \sum_{f \in \mathcal{F}_p} \phi(\boldsymbol{x}_{pf},t) a_{pf} =
	\sum_{f \in \mathcal{F}_p} \int_{t^{n-1}}^{t^n} \phi(\boldsymbol{x}_{pf},t) a_{pf}
	\approx
	\Delta t \sum_{f \in \mathcal{F}_p} \phi(\boldsymbol{x}_{pf},t^{n-1/2}) a_{pf},
\]
where we approximate the time integral of $ \phi $ using the midpoint rule. \\
The discretization of \eqref{eqn:scalar-conservation-law-3d} thus reads as
\begin{equation}
	\frac{|\Omega_p|}{\Delta t}(\phi_p^n - \phi_p^{n-1})
	+ \sum_{f \in \mathcal{F}_p} \phi(\boldsymbol{x}_{pf},t^{n-1/2}) a_{pf} = 0.
\end{equation}
Next we approximate the value at the face center at half-time step
\[
	\phi(\boldsymbol{x}_{pf},t^{n-1/2})
	\approx
	\frac{1}{2}(\phi(\boldsymbol{x}_{pf},t^n) + \phi(\boldsymbol{x}_{pf},t^{n-1}))
	\equiv
	\phi_{f}^{n-1/2}.
\]
For the outflow case $ a_{pf} \geq 0 $
\[
	\phi_{f}^{n-1/2} = \frac{1}{2}(\phi_{pf}^n + \phi_{qf}^{n-1}),
\]
where
\[
	\phi_{pf}^{n} = \phi_p^{n} + \nabla\phi_p^{n}\cdot\boldsymbol{d}_{pf},
	\quad \text{and} \quad
	\phi_{qf}^{n-1} = \phi_q^{n-1} + \nabla\phi_q^{n-1}\cdot\boldsymbol{d}_{qf}.
\]
We take the average of reconstructed face values from neighboring cells,
the upwind value is treated implicitly, and the downwind explicitly.
This mimics the approximation of the face value of the IIOE method in 1D:
\[
	\phi_{i+1/2}^{n-1/2} = \frac{1}{2}(\phi_i^{n} + \phi_{i+1}^{n-1}).
\]
For the outflow case $ a_{pf} < 0 $
\[
	\phi_{f}^{n-1/2} = \frac{1}{2}(\phi_{pf}^{n-1} + \phi_{qf}^{n}),
\]
where
\begin{equation}
	\phi_{pf}^{n-1} = \phi_p^{n-1} + \nabla\phi_p^{n-1}\cdot\boldsymbol{d}_{pf},
	\quad \text{and} \quad
	\phi_{qf}^{n} = \phi_q^{n} + \nabla\phi_q^{n}\cdot\boldsymbol{d}_{qf}.
\end{equation}
We divide the set of faces to inflow and outflow parts
\begin{align*}
	\mathcal{B}_p^- &= \left\{b \in \mathcal{B}_p\, |\, a_{pb} < 0\right\}
	& &\text{and} &
	\mathcal{B}_p^+ &= \mathcal{B}_p \setminus \mathcal{B}_p^- \\
	\mathcal{F}_p^- &= \left\{f \in (\mathcal{F}_p \setminus \mathcal{B}_p) \, |\, a_{pf} < 0\right\}
	& &\text{and} &
	\mathcal{F}_p^+ &= (\mathcal{F}_p \setminus \mathcal{B}_p) \setminus \mathcal{F}_p^-.
\end{align*}
The discretization of \eqref{eqn:scalar-conservation-law-3d} takes the form
\begin{equation}\label{eqn:fvm-discretization-3d}
	\begin{split}
	\frac{|\Omega_p|}{\Delta t}(\phi_p^{n} - \phi_p^{n-1})
	&+ \sum_{f \in \mathcal{F}_p^-} \phi_{pf}^{n-1/2} a_{pf}
	+ \sum_{f \in \mathcal{F}_p^+} \phi_{pf}^{n-1/2} a_{pf}\\
	&+ \sum_{f \in \mathcal{B}_p^-} \phi_{pb}^{n-1/2} a_{pb}
	+ \sum_{f \in \mathcal{B}_p^+} \phi_{pb}^{n-1/2} a_{pb} = 0,
	\end{split}
\end{equation}
and the face values depend on the sign of flux $ a_{pf} $
\[
	\phi_{pf}^{n-1/2} =
	\begin{cases}
		\frac{1}{2}(\phi_p^n + \nabla\phi_p^n\cdot\boldsymbol{d}_{pf} +
		\phi_q^{n-1} + \nabla\phi_q^{n-1}\cdot\boldsymbol{d}_{qf})
		& \text{if}\ a_{pf} \geq 0 \\
		\frac{1}{2}(\phi_p^{n-1} + \nabla\phi_p^{n-1}\cdot\boldsymbol{d}_{pf} +
		\phi_q^{n} + \nabla\phi_q^{n}\cdot\boldsymbol{d}_{qf})
		& \text{if}\ a_{pf} < 0,
	\end{cases}
\]
\[
	\phi_{pb}^{n-1/2} =
	\begin{cases}
		\frac{1}{2}\left(\phi_p^n + \nabla\phi_p^n\cdot\boldsymbol{d}_{pf} +
		\phi_e(\boldsymbol{x}_{pb}, t^{n-1})\right)
		& \text{if}\ a_{pf} \geq 0 \\
		\frac{1}{2}
		\left(
			\phi_e(\boldsymbol{x}_{pb}, t^{n-1})
			+ \phi_e(\boldsymbol{x}_{pb}, t^{n})
		\right)
		& \text{if}\ a_{pf} < 0.
	\end{cases}
\]
For the case $ a_{pf} \geq 0 $ we can rewrite the approximation of the face values as
\begin{equation}
	\begin{split}
		\phi_{pf}^{n-1/2}
		&= \frac{1}{2}(\phi_p^n + \nabla\phi_p^n\cdot\boldsymbol{d}_{pf} +
		\phi_q^{n-1} + \nabla\phi_q^{n-1}\cdot\boldsymbol{d}_{qf}) \\
		&= \phi_p^n + \frac{1}{2}(-\phi_p^n + \nabla\phi_p^n\cdot\boldsymbol{d}_{pf} +
		\phi_q^{n-1} + \nabla\phi_q^{n-1}\cdot\boldsymbol{d}_{qf}),
	\end{split}
\end{equation}
which is now written as the upwind cell center value plus a correction term.
To simplify the notation, we denote the correction term as
\[
	\mathcal{D}^+_{pf}\phi
	\equiv
	\frac{1}{2}(-\phi_p^n + \nabla\phi_p^n\cdot\boldsymbol{d}_{pf} +
	\phi_q^{n-1} + \nabla\phi_q^{n-1}\cdot\boldsymbol{d}_{qf}),
\]
the plus sign indicating that $ a_{pf} \geq 0 $ is non-negative.
To ensure boundedness, we apply a limiter $ \Psi_{f} $ to the correction term
\begin{equation} \label{rec_face}
	\phi_{pf}^{n-1/2} = \phi_p^n + \Psi_{f} \mathcal{D}^+_{pf}\phi
\end{equation}
For the case $ a_{pf} < 0 $ we can rewrite the approximation of the face values as
\begin{equation}
	\begin{split}
		\phi_{pf}^{n-1/2}
		&= \frac{1}{2}(\phi_q^{n} + \nabla\phi_q^{n}\cdot\boldsymbol{d}_{qf} +
		\phi_p^{n-1} + \nabla\phi_p^{n-1}\cdot\boldsymbol{d}_{pf}) \\
		&= \phi_q^{n} + \frac{1}{2}(-\phi_q^{n} + \nabla\phi_q^{n}\cdot\boldsymbol{d}_{qf} +
		\phi_p^{n-1} + \nabla\phi_p^{n-1}\cdot\boldsymbol{d}_{pf}),
	\end{split}
\end{equation}
which is now written, as in the outflow case,
as the upwind cell center value plus a correction term.
To simplify the notation, we denote the correction term as
\[
	\mathcal{D}_{pf}^-\phi
	\equiv
	\frac{1}{2}(-\phi_q^{n} + \nabla\phi_q^{n}\cdot\boldsymbol{d}_{qf} +
	\phi_p^{n-1} + \nabla\phi_p^{n-1}\cdot\boldsymbol{d}_{pf}),
\]
the minus sign indicating that $ a_{pf} < 0 $ is negative.

We do similarly at boundary faces. For \(a_{pb} \geq 0\) we have
\begin{align*}
	\phi_{pb}^{n-1/2}
	&= \frac{1}{2}
	\left(
		\phi_p^{n} + \nabla\phi_p^{n}\cdot\boldsymbol{d}_{pb} + \phi_e(\boldsymbol{x}_{pb}, t^{n-1})
	\right)\\
	&= \phi_p^{n} + \frac{1}{2}
	\left(
		-\phi_p^{n} + \nabla\phi_p^{n}\cdot\boldsymbol{d}_{pb} + \phi_e(\boldsymbol{x}_{pb}, t^{n-1})
	\right)\\
	&= \phi_p^{n} + \mathcal{D}^+_{pb}\phi,
\end{align*}
where \[
	\mathcal{D}^+_{pb}\phi =
	\frac{1}{2}\left(
		-\phi_p^n + \nabla\phi_p^n\cdot\boldsymbol{d}_{pf} + \phi_e(\boldsymbol{x}_{pb}, t^{n-1})
		\right).
	\]
For \(a_{pb} < 0\) we have
\begin{align*}
	\phi_{pb}^{n-1/2}
	&= \frac{1}{2}
	\left(
		\phi_p^{n-1} + \nabla\phi_p^{n-1}\cdot\boldsymbol{d}_{pb} + \phi_e(\boldsymbol{x}_{pb}, t^{n})
	\right)\\
	&= \phi_e(\boldsymbol{x}_{pb}, t^{n}) + \frac{1}{2}
	\left(
		\phi_p^{n-1} + \nabla\phi_p^{n-1}\cdot\boldsymbol{d}_{pb} - \phi_e(\boldsymbol{x}_{pb}, t^{n})
	\right)\\
	&= \phi_e(\boldsymbol{x}_{pb}, t^{n}) + \mathcal{D}^-_{pb}\phi,
\end{align*}
We end up with a linear system of equations
\begin{equation}
	\begin{split}
		\frac{|\Omega_p|}{\Delta t}(\phi_p^{n} - \phi_p^{n-1})
	&+ \sum_{f \in \mathcal{F}_p^-}
	\left(
		\phi_q^n + \frac{\Psi_f}{2}
		\left(
			-\phi_q^n + \nabla\phi_q^n\cdot\boldsymbol{d}_{qf} +
			\phi_p^{n-1} + \nabla\phi_p^{n-1}\cdot\boldsymbol{d}_{pf}
		\right)
	\right) a_{pf}\\
	&+ \sum_{f \in \mathcal{F}_p^+}
	\left(
		\phi_p^n + \frac{\Psi_f}{2}
		\left(
			-\phi_p^n + \nabla\phi_p^n\cdot\boldsymbol{d}_{pf} +
			\phi_q^{n-1} + \nabla\phi_q^{n-1}\cdot\boldsymbol{d}_{qf}
		\right)
	\right) a_{pf}\\
	&+ \sum_{f \in \mathcal{B}_p^-}
	\left(
		\phi_e(\boldsymbol{x}_{pb}, t^{n}) + \frac{\Psi_b}{2}
		\left(
			\phi_p^{n-1} + \nabla\phi_p^{n-1}\cdot\boldsymbol{d}_{pb}
			- \phi_e(\boldsymbol{x}_{pb}, t^{n})
		\right)
	\right) a_{pb}\\
	&+ \sum_{f \in \mathcal{B}_p^+}
	\left(
		\phi_p^{n} + \frac{\Psi_b}{2}
		\left(
			-\phi_p^{n} + \nabla\phi_p^{n}\cdot\boldsymbol{d}_{pb}
			+ \phi_e(\boldsymbol{x}_{pb}, t^{n-1})
		\right)
	\right) a_{pb} = 0.
	\end{split}
\end{equation}
We solve the system iteratively by applying the deferred correction procedure~\cite{2016_Moukalled_BOOK} for similar reasons as in~\cite{2019_Hahn}.
In order to keep the parallel computations simple, in each cell a 1-ring neighborhood structure should be considered only.
However, to compute the gradient of a neighbor cell \(q\), we would need access to cells outside the 1-ring neighborhood. Thus, the gradients of neighbors \( q \) should be computed using the latest available values from the previous iteration.
<for the case \(a_{pf} \geq 0\), the face value in the \(k\)-th iteration becomes
\begin{equation}
	\phi_{pf}^{n-1/2,k} = \phi_p^{n,k} + \Psi_{f}^{k-1} \mathcal{D}^{+,k-1}_{pf}\phi
\end{equation}
\begin{equation}
	\begin{split}
		\frac{|\Omega_p|}{\Delta t}(\phi^{n,k}_p - \phi_p^{n-1})
		&+ \sum_{f \in \mathcal{F}_p^-}
		\left(
			\phi_q^{n,k} +\Psi_{pf}^{-, k-1}\mathcal{D}^-_{pf}\phi^{k-1}
		\right) a_{pf}\\
		&+ \sum_{f \in \mathcal{F}_p^+}
		\left(
			\phi_p^{n,k} + \Psi_{pf}^{+, k-1}\mathcal{D}^+_{pf}\phi^{k-1}
		\right) a_{pf}\\
		&+ \sum_{b \in \mathcal{B}_p^-}
		\left(
			\phi_e(\boldsymbol{x}_{pb}, t^{n})
			+ \Psi_{pb}^-\mathcal{D}^-_{pb}\phi
		\right) a_{pb}\\
		&+ \sum_{b \in \mathcal{B}_p^+}
		\left(\phi_p^{n,k} +
		\Psi_{pb}^{+, k-1}\mathcal{D}^+_{pb}\phi^{k-1}\right) a_{pb} = 0.
	\end{split}
\end{equation}
This yields a system with an M-matrix, which has favorable properties (\textcolor{red}{reference here}). The inverse of an M-matrix has positive components. We can think of the solution of the system as applying the inverse matrix \(\mathbf{M}^{-1}\) to the RHS.

We rearrange the terms
\begin{equation}
	\begin{split}
		\frac{|\Omega_p|}{\Delta t}\phi^{n,k}_p
		&+ \sum_{f \in \mathcal{F}_p^-}
		\phi_q^{n,k} a_{pf}
		+ \sum_{f \in \mathcal{F}_p^+}
		\phi_p^{n,k} a_{pf}
		+ \sum_{b \in \mathcal{B}_p^+}
		\phi_p^{n,k} a_{pb}
		=\\
		\frac{|\Omega_p|}{\Delta t}\phi_p^{n-1}
		&-\sum_{f \in \mathcal{F}_p^-}
		\Psi_{pf}^{-, k-1}\mathcal{D}^-_{pf}\phi^{k-1}  a_{pf}
		- \sum_{f \in \mathcal{F}_p^+}
		\Psi_{pf}^{+, k-1}\mathcal{D}^+_{pf}\phi^{k-1}  a_{pf}
		\\
		&-\sum_{b \in \mathcal{B}_p^-}
		\left(
			\phi_e(\boldsymbol{x}_{pb}, t^{n})
			+ \Psi_{pb}^-\mathcal{D}^-_{pb}\phi
		\right) a_{pb}
		- \sum_{b \in \mathcal{B}_p^+}
		\Psi_{pb}^{+, k-1}\mathcal{D}^+_{pb}\phi^{k-1} a_{pb}.
	\end{split}
\end{equation}
For internal cells, we require for the right-hand side
\begin{equation*}
	\begin{split}
		\phi_{p, vnbd}^{min}
		\leq
		\frac{|\Omega_p|}{\Delta t}\phi_p^{n-1}
		- \sum_{f \in \mathcal{F}_p^-}
		\Psi_{pf}^{-,k-1}\mathcal{D}^-_{pf}\phi^{k-1} a_{pf}
		- \sum_{f \in \mathcal{F}_p^+}
		\Psi_{pf}^{+,k-1}\mathcal{D}^+_{pf}\phi^{k-1} a_{pf}
		\leq
		\phi_{p, vnbd}^{max},
	\end{split}
\end{equation*}
\end{document}