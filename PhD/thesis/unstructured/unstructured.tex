\documentclass[../thesis.tex]{subfiles}
\begin{document}
Let \(\Omega \subset \mathbb{R}^3 \) be a bounded Lipschitz
domain of our interest, and \(\partial \Omega\) its boundary. In this work,
our aim is to describe a numerical approximation of the scalar conservation
law, also known as the continuity equation
\begin{equation}\label{eqn:scalar-conservation-law-3d}
	\phi_t
	+ \nabla \cdot \left( \phi \mathbf{u} \right) = 0,
\end{equation}
where \(\phi = \phi(\mathbf{x},t)\) is the conserved scalar quantity,
\(\left( \mathbf{x}, t \right) \in \Omega \times \left[ 0,T \right]\).
For simplicity, we consider velocity fields
\(\mathbf{u} = \mathbf{u}(\mathbf{x})\) that are functions of~the
position only.
Initial and Dirichlet boundary conditions are given
\begin{align}
	\phi(\mathbf{x},0) &= \phi^0(\mathbf{x}), \quad \mathbf{x} \in \Omega\\
	\phi(\mathbf{x}_b,t) &= \phi_b(t), \quad (\mathbf{x}, t) \in \Omega \times
	\left[ 0,T \right].
\end{align}
\subsection{Discretization of the domain}
In case of a complex computational domain, it is often
necessary to use unstructured polyhedral mesh.
The finite volume method (FVM) is well suited for space-discretization of
arbitrary shape. In FVM, the unknowns are the cell averages.
A particular finite volume scheme is defined by its approximation of
the fluxes at cell faces.
% \subsection{Distorted cell and subface tessellation}
Let p be a finite volume cell. We discretize the domain \(\Omega\)
by open non-overlapping polyhedral sets \(\Omega_p \subset \Omega\).
Let \(\bar{\Omega} =
\bigcup_{p \in \mathcal{I}} \bar{\Omega}_p \subset \mathbb{R}^3\)
be a computational domain, where an open set \(\Omega_p\) with the non-zero volume
\(|\Omega_p|\) is a cell in discretized domain and \(\mathcal{I}\) is the set
of cell indices, \(p \in \mathcal{I}\).

A distorted cell means that the cell has a nonplanar face,
which usually exists when there is a complex computation boundary. It is an
unrealistic assumption that all faces of \(|\Omega_p|\) for all
\(p \in \mathcal{I}\) are of a planar shape in a case of industrial problems.

Whenever a face is not triangle, the face is tessellated into triangles;
from a given center of the face, two consecutive vertices of the face
are selected to construct one triangle. A center \(\mathbf{x}^*\)
of face whose vertices are \(\mathbf{x}_i, i = 1 , \dots, r\) is
usually given by the area average. To make it simple, using a cyclic
notation \(\mathbf{x}_{r + 1} = \mathbf{x}_1 \), the area
average face center is computed by
\begin{equation}
	\mathbf{x}^* =
	\frac{\sum_{i=1}^r |\Delta_i| \bar{\mathbf{x}}_i}{\sum_{i=1}^{r} |\Delta_i|}
\end{equation}
where \(|\Delta_i|\) is the area of \(\Delta_i\), \(\bar{\mathbf{x}}_i\)
is the center of mass on \(\Delta_i\), and \(\Delta_i\) is a triangle of
\(\mathbf{x}_i,\, \mathbf{x}_{i+1}\), and
\(\mathbf{x}_0 = \frac{1}{r}\sum_{j=1}^r \mathbf{x}_j\)
for \(i = 1,\dots,r\). Then, the \(i\)-th tessellated triangle at
the face is defined by three points:
\(\mathbf{x}_i, \mathbf{x}_{i+1},\text{ and } \mathbf{x}^*\).
The index set \(\mathcal{F}\) denotes all
triangles \(e_f\), \(f \in \mathcal{F}\) tessellated from a face or a triangle
face of \(\Omega_p\) for all \(p \in \mathcal{I} \). We denote \(\mathbf{x}_f\)
as the center of triangle \(e_f\).

To indicate the neighbor cells of \(\Omega_p\) we consider only the cells
whose face is shared:
\[
	\mathcal{N}_p = \left\{
		q \in \mathcal{I}:\text{ there exists a face }
		e_f \subset \partial\Omega_q \cap \partial\Omega_p, f \in \mathcal{F}
	\right\}.
\]
The faces of \(\Omega_p\) are indicated by two sets:
\[
	\mathcal{F}_p = \left\{
		f \in \mathcal{F}: e_f \subset \partial \Omega_p
	\right\}\quad
	\text{and}\quad
	\mathcal{B}_p = \left\{
		f \in \mathcal{F}: e_f \subset \partial \Omega_p \cap \partial \Omega
	\right\}.
\]
If \(\Omega_p\) is a cell who's all faces are not overlapped to \(\partial \Omega\),
then we call the cell as an internal cell and \(\mathcal{B}_p = \emptyset\).
Otherwise, we call the cell as a boundary cell. Similarly, an internal face \(e_f\),
\(f \in \mathcal{F}\), means that the face is not overlapped to \(\partial\Omega\)
and a boundary face \(e_b\), \(b \in \mathcal{F}\), is a part of \(\partial \Omega\).
Throughout the rest of paper, the subscript \(b\) indicates the face index whose face
is a boundary face. When a quantity is defined on a face \(e_f\), and it depends on
a cell sharing the face, the cell index should be explicitly indicated at the first
subscript and the face index at the second subscript. For instance, for a face \(e_f \)
\(f \in \mathcal{F}_p\), the outward normal vector to the face is indicated by
\(\mathbf{n}_{pf}\). Note that we also use the length of the normal vector as the area of
the face, that is, \(|\mathbf{n}_{pf}| = |e_f|\). If a face \(e_f\) is an internal face
for \(f \in \mathcal{F}_p\), there exists a cell \(\Omega_p,\, q\in \mathcal{I}\),
such that \(e_f \subset \partial\Omega_p \cap \Omega_q\).
Then, clearly, \(\mathbf{n}_{pf} = -\mathbf{n}_{qf}\).
Whenever a directional vector is denoted, we use a notation \(\mathbf{d}\)
with relevant indices:
\[
	\mathbf{d}_{pq} = \mathbf{x}_q - \mathbf{x}_p,\quad
	p,q \in \mathcal{I}.
\]
\subsection{Numerical scheme}
We integrate \eqref{eqn:scalar-conservation-law-3d} in space and time
\begin{equation}\label{eqn:integral-conservation-law-3d}
	\int_{t^{n-1}}^{t^n} \int_{\Omega_p}
	\phi_t
	+\int_{t^{n-1}}^{t^n} \int_{\Omega_p}
	\nabla \cdot \left( \phi \mathbf{u}\right) = 0.
\end{equation}
In the first term we change the order of integration
and apply the Newton-Leibniz theorem for the time integration
\[
	\int_{t^{n-1}}^{t^n} \int_{\Omega_p}
	\phi_t =
	\int_{\Omega_p} \int_{t^{n-1}}^{t^n}
	\phi_t =
	\int_{\Omega_p}
	\left(\phi^n(\mathbf{x}) - \phi^{n-1}(\mathbf{x})\right)
	= |\Omega_p|(\overline{\phi}^{n}_p - \overline{\phi}_p^{n-1}),
\]
where we denote \(\phi(\mathbf{x}, t^n) = \phi^n(\mathbf{x})\),
and, \(\overline{\phi}_p^{n}\) is the cell average at time \(t^n\) such that
\begin{equation}\label{eqn:cell-avg-integral}
	\overline{\phi}^{n}_p
	= \frac{1}{|\Omega_p|}\int_{\Omega_p} \phi^{n}(\mathbf{x}).
\end{equation}
Next we apply
the divergence theorem to the volume integral
\[
	\int_{\Omega_p}
	\nabla \cdot \left(\phi\mathbf{u}\right) =
	\int_{\partial\Omega_p}
	\left( \phi\mathbf{u} \right) \cdot
	\frac{\mathbf{n}_p}{|\mathbf{n}_p|} =
	\sum_{f \in \mathcal{F}_p} \int_{e_f}
	\left( \phi\mathbf{u} \right)\cdot
	\frac{\mathbf{n}_{pf}}{|\mathbf{n}_{pf}|},
\]
where \(\mathbf{n}_{pf}\) is an outward facing normal vector of
the face \(e_f\).
Thus, we can rewrite the integral form of the conservation law
\eqref{eqn:integral-conservation-law-3d} as
\begin{equation}\label{eqn:integral-conservation-law-cell-3d}
	|\Omega_p|(\overline{\phi}_p^{n} - \overline{\phi}_p^{n-1})
	+ \int_{t^{n-1}}^{t^n} \sum_{f \in \mathcal{F}_p} \int_{e_f}
	\left( \phi\mathbf{u} \right)\cdot
	\frac{\mathbf{n}_{pf}}{|\mathbf{n}_{pf}|} = 0,
\end{equation}
which is still exact for a finite volume cell \(p\).

Throughout this text, we will approximate the cell average with the cell
center value
\[
	\overline{\phi}_p^{n} \approx \phi_p^{n}.
\]
In order to show that this is a second order approximation,
we describe the spatial variation of \(\phi\) within an element
via Taylor series expansion around the cell center \(\mathbf{x}_p\) as
\begin{equation}\label{eqn:taylor-3d}
	\phi^n(\mathbf{x}) = \phi_p^{n}
	+ \nabla \phi_p^{n} \cdot \left( \mathbf{x} - \mathbf{x}_p \right)
	+ \mathcal{O}\left( |\mathbf{x} - \mathbf{x}_p|^2 \right).
\end{equation}
We substitute
\eqref{eqn:taylor-3d} to \eqref{eqn:cell-avg-integral}
\begin{equation}\label{eqn:cell-avg-proof}
	\begin{split}
		\overline{\phi}^{n}_p
		&= \frac{1}{|\Omega_p|}\int_{\Omega_p} \phi^{n}(\mathbf{x})\\
		&= \frac{1}{|\Omega_p|}\int_{\Omega_p} \left[ \phi_p^{n}
		+ \nabla \phi_p^{n} \cdot \left( \mathbf{x} - \mathbf{x}_p \right)
		+ \mathcal{O}\left( |\mathbf{x} - \mathbf{x}_p|^2 \right) \right]\\
		&= \frac{1}{|\Omega_p|}\int_{\Omega_p} \phi_p^{n}
		+ \frac{1}{|\Omega_p|}\int_{\Omega_p} \nabla \phi_p^{n} \cdot
		\left( \mathbf{x} - \mathbf{x}_p \right)
		+ \frac{1}{|\Omega_p|}\int_{\Omega_p} \mathcal{O}\left( |\mathbf{x} - \mathbf{x}_p|^2 \right)\\
		&= \phi_p^{n} + \frac{1}{|\Omega_p|}\int_{\Omega_p} \mathcal{O}\left( |\mathbf{x} - \mathbf{x}_p|^2 \right).
	\end{split}
\end{equation}
The second integral equals to~0, since \(\mathbf{x}_p\) is the cell center.
For more details, see, e.g.,~\cite{2016_Moukalled_BOOK,1996_Jasak}.

We approximate the expression under the integral
with the face center values
\[
	\int_{e_f}
	\left( \phi\mathbf{u} \right)\cdot
	\frac{\mathbf{n}_{pf}}{|\mathbf{n}_{pf}|}
	\approx
	\int_{e_f} \phi(\mathbf{x}_{pf},t) \mathbf{u}(\mathbf{x}_{pf}) \cdot
	\frac{\mathbf{n}_{pf}}{|\mathbf{n}_{pf}|}
	= \phi(\mathbf{x}_{pf},t) \mathbf{u}(\mathbf{x}_{pf}) \cdot \mathbf{n}_{pf},
\]
since we defined the normal vector such that \(|\mathbf{n}_{pf}| = |e_f|\).
Similarly to~\eqref{eqn:cell-avg-proof}, it can be shown
that this is a second order approximation of the face integral.
See, e.g.,~\cite{2016_Moukalled_BOOK,1996_Jasak}.

We denote the volumetric flux as
\[
	a_{pf} \equiv \mathbf{u}(\mathbf{x}_{pf}) \cdot \mathbf{n}_{pf}.
\]
Thus, we end up with the approximation
of the divergence term
\[
	\int_{\Omega_p}
	\nabla \cdot \left(\phi\mathbf{u}\right)
	\approx
	\sum_{f \in \mathcal{F}_p} \phi(\mathbf{x}_{pf},t) a_{pf}.
\]
In order to achieve second order in time, we approximate
the time integral of the fluxes using the midpoint rule
\[
	\int_{t^{n-1}}^{t^n} \sum_{f \in \mathcal{F}_p} \phi(\mathbf{x}_{pf},t) a_{pf}
	\approx
	\Delta t \sum_{f \in \mathcal{F}_p}
	\phi(\mathbf{x}_{pf},t^{n-1/2}) a_{pf}
	=\Delta t \sum_{f \in \mathcal{F}_p}
	\phi_{pf}^{n-1/2} a_{pf},
\]
where \(\Delta t = t^n - t^{n-1}\), is the uniform time step,
\(t^{n-1/2} = \frac{t^n+t^{n-1}}{2}\), and we use the notation
\(\phi(\mathbf{x}_{pf},t^{n}) = \phi_{pf}^{n}\).

Thus, the second order in space and time discretization
of~\eqref{eqn:integral-conservation-law-cell-3d} reads as
\begin{equation}\label{eqn:integral-claw-2ndO-approx-3d}
	|\Omega_p|(\phi_p^n - \phi_p^{n-1})
	+ \Delta t \sum_{f \in \mathcal{F}_p}
	\phi_{pf}^{n-1/2} a_{pf} = 0.
\end{equation}
Next we express the value at the face center at half-time step
as the average value of old and new values. We take the average
of reconstructed face values from neighboring cells,
the upwind value is treated implicitly, and the downwind explicitly.
In particular, for the outflow case \(a_{pf} \geq 0\)
\[
	\phi_{pf}^{n-1/2} = \frac{1}{2}(\phi_{pf}^n + \phi_{qf}^{n-1}),
\]
where
\[
	\phi_{pf}^{n} = \phi_p^{n} + \nabla\phi_p^{n}\cdot\mathbf{d}_{pf},
	\quad \text{and} \quad
	\phi_{qf}^{n-1} = \phi_q^{n-1} + \nabla\phi_q^{n-1}\cdot\mathbf{d}_{qf}.
\]
This mimics the approximation of the face value of the IIOE method
\cite{2014_Mikula,2019_Hahn,2020_Ibolya_CONF} in 1D:
\[
	\phi_{i+1/2}^{n-1/2} = \frac{1}{2}(\phi_i^{n} + \phi_{i+1}^{n-1}).
\]
For the inflow case \(a_{pf} < 0\)
\[
	\phi_{f}^{n-1/2} = \frac{1}{2}(\phi_{pf}^{n-1} + \phi_{qf}^{n}),
\]
where
\begin{equation}
	\phi_{pf}^{n-1} = \phi_p^{n-1} + \nabla\phi_p^{n-1}\cdot\mathbf{d}_{pf},
	\quad \text{and} \quad
	\phi_{qf}^{n} = \phi_q^{n} + \nabla\phi_q^{n}\cdot\mathbf{d}_{qf}.
\end{equation}
We divide the set of faces to inflow and outflow parts
\begin{align*}
	\mathcal{B}_p^- &= \left\{b \in \mathcal{B}_p\, |\, a_{pb} < 0\right\}
	& &\text{and} &
	\mathcal{B}_p^+ &= \mathcal{B}_p \setminus \mathcal{B}_p^- \\
	\mathcal{F}_p^- &= \left\{f \in (\mathcal{F}_p \setminus \mathcal{B}_p) \, |\, a_{pf} < 0\right\}
	& &\text{and} &
	\mathcal{F}_p^+ &= (\mathcal{F}_p \setminus \mathcal{B}_p) \setminus \mathcal{F}_p^-.
\end{align*}
Thus, we can rewrite \eqref{eqn:integral-claw-2ndO-approx-3d}
\begin{equation}\label{eqn:fvm-discretization-3d}
	\begin{split}
	\frac{|\Omega_p|}{\Delta t}(\phi_p^{n} - \phi_p^{n-1})
	&+ \sum_{f \in \mathcal{F}_p^-} \phi_{pf}^{n-1/2} a_{pf}
	+ \sum_{f \in \mathcal{F}_p^+} \phi_{pf}^{n-1/2} a_{pf}\\
	&+ \sum_{b \in \mathcal{B}_p^-} \phi_{pb}^{n-1/2} a_{pb}
	+ \sum_{b \in \mathcal{B}_p^+} \phi_{pb}^{n-1/2} a_{pb} = 0,
	\end{split}
\end{equation}
where the face values depend on the sign of flux \(a_{pf}\)
\begin{equation}\label{eqn:faceval-no-limiter}
	\phi_{pf}^{n-1/2} =
	\begin{cases}
		\frac{1}{2}(\phi_p^n + \nabla\phi_p^n\cdot\mathbf{d}_{pf} +
		\phi_q^{n-1} + \nabla\phi_q^{n-1}\cdot\mathbf{d}_{qf})
		& \text{if}\ a_{pf} \geq 0 \\
		\frac{1}{2}(\phi_p^{n-1} + \nabla\phi_p^{n-1}\cdot\mathbf{d}_{pf} +
		\phi_q^{n} + \nabla\phi_q^{n}\cdot\mathbf{d}_{qf})
		& \text{if}\ a_{pf} < 0,
	\end{cases}
\end{equation}
\begin{equation}\label{eqn:faceval-bdry-no-limiter}
	\phi_{pb}^{n-1/2} =
	\begin{cases}
		\frac{1}{2}\left(\phi_p^n + \nabla\phi_p^n\cdot\mathbf{d}_{pf} +
		\phi_e(\mathbf{x}_{pb}, t^{n-1})\right)
		& \text{if}\ a_{pb} \geq 0 \\
		\frac{1}{2}
		\left(
			\phi_p^{n-1} + \nabla\phi_p^{n-1}\cdot\mathbf{d}_{pf}
			+ \phi_e(\mathbf{x}_{pb}, t^{n})
		\right)
		& \text{if}\ a_{pb} < 0.
	\end{cases}
\end{equation}
Substituting \eqref{eqn:faceval-no-limiter}, \eqref{eqn:faceval-bdry-no-limiter}
to \eqref{eqn:integral-claw-2ndO-approx-3d} yields a linear system of equations
\begin{equation}
	\begin{split}
		\frac{|\Omega_p|}{\Delta t}(\phi_p^{n} - \phi_p^{n-1})
	&+ \sum_{f \in \mathcal{F}_p^-}
	\frac{1}{2}\left(
			\phi_p^{n-1} + \nabla\phi_p^{n-1}\cdot\mathbf{d}_{pf} +
			\phi_q^{n} + \nabla\phi_q^{n}\cdot\mathbf{d}_{qf}
		\right) a_{pf}\\
	&+ \sum_{f \in \mathcal{F}_p^+}
	\frac{1}{2}\left(
			\phi_p^n + \nabla\phi_p^n\cdot\mathbf{d}_{pf} +
			\phi_q^{n-1} + \nabla\phi_q^{n-1}\cdot\mathbf{d}_{qf}
		\right) a_{pf}\\
	&+ \sum_{b \in \mathcal{B}_p^-}
	\frac{1}{2}\left(
			\phi_p^{n-1} + \nabla\phi_p^{n-1}\cdot\mathbf{d}_{pf}
			+ \phi_e(\mathbf{x}_{pb}, t^{n})
		\right) a_{pb}\\
	&+ \sum_{b \in \mathcal{B}_p^+}
	\frac{1}{2}\left(
		\phi_p^n + \nabla\phi_p^n\cdot\mathbf{d}_{pf} +
		\phi_e(\mathbf{x}_{pb}, t^{n-1})
		\right) a_{pb} = 0.
	\end{split}
\end{equation}
As described in \cite{2019_Hahn}, because of practical considerations,
the simplest overlapping domain decomposition is required under the 1-ring face
neighborhood structure in order to simplify the parallel computations. From a
cell it is possible to access the values from neighboring cells. However,
because of the unknown gradient terms of the neighbors \(\nabla\phi_q^{n}\),
we would need values from a 2-ring neighborhood. Another complication is that
for the gradient term \(\nabla\phi_p^n\) it is quite cumbersome to extract
the coefficients needed for the neighbors. So, because of the above practical
considerations, we employ a deferred correction procedure
\cite{1974_Khosla,2016_Moukalled_BOOK,2019_Hahn} to solve the system.
The method we use is based on writing a high-order flux terms
\eqref{eqn:faceval-no-limiter}, \eqref{eqn:faceval-bdry-no-limiter} - second order
in our case - as a low-order scheme plus a high-order correction term as
\begin{equation}\label{eqn:high-order-low-order-combination}
	\phi_{pf}^{HO} = \phi_{pf}^{LO} + (\phi_{pf}^{HO} - \phi_{pf}^{LO}).
\end{equation}
In our case of an implicit scheme, the first term appears in our calculations
implicitly, while the second term is evaluated using the latest available values.
By choosing an appropriate low-order flux, the linear-system will have an
M-matrix, which has favorable stability properties (\textcolor{red}{reference here}).

For the case \(a_{pf} \geq 0\) we can rewrite the approximation of the face values as
\begin{equation}
	\begin{split}
		\phi_{pf}^{n-1/2}
		&= \frac{1}{2}(\phi_p^n + \nabla\phi_p^n\cdot\mathbf{d}_{pf} +
		\phi_q^{n-1} + \nabla\phi_q^{n-1}\cdot\mathbf{d}_{qf}) \\
		&= \phi_p^n + \frac{1}{2}(-\phi_p^n + \nabla\phi_p^n\cdot\mathbf{d}_{pf} +
		\phi_q^{n-1} + \nabla\phi_q^{n-1}\cdot\mathbf{d}_{qf}),
	\end{split}
\end{equation}
which is now written as \eqref{eqn:high-order-low-order-combination}, where the low-order
flux is the implicit upwind flux
\begin{equation}\label{eqn:faceval-implicit-upwind}
	\phi_{pf}^{LO} =
	\begin{cases}
		\phi_p^n
		& \text{if}\ a_{pf} \geq 0 \\
		\phi_q^{n}
		& \text{if}\ a_{pf} < 0.
	\end{cases}
\end{equation}
By applying the deferred correction procedure
discussed above, the face value in the \(k\)-th iteration is computed as
\begin{equation}\label{eqn:fval-iteration-outflow}
	\phi_{pf}^{n-1/2,k} = \phi_p^{n,k} +
	\frac{1}{2}(-\phi_p^{n,k-1} + \nabla\phi_p^{n,k-1}\cdot\mathbf{d}_{pf} +
	\phi_q^{n-1} + \nabla\phi_q^{n-1}\cdot\mathbf{d}_{qf}),
\end{equation}
where \(k = 1 \dots K\), \(\phi_p^{n,0} = \phi_p^{n-1}\) and
\(\nabla\phi_p^{n,k-1}\) is computed using \(\phi_p^{n,k-1}\).
To simplify the notation, we denote the correction term as
\begin{equation}\label{eqn:correction-term-outflow}
	\mathcal{D}^+_{pf}\phi^{k}
	\equiv
	\frac{1}{2}(-\phi_p^{n,k} + \nabla\phi_p^{n,k}\cdot\mathbf{d}_{pf} +
	\phi_q^{n-1} + \nabla\phi_q^{n-1}\cdot\mathbf{d}_{qf}),
\end{equation}
where the plus sign indicates that \(a_{pf} \geq 0 \) is non-negative.
Analogously to the outflow case, if \(a_{pf} < 0\), we can rewrite the
approximation of the face values as
\begin{equation}\label{eqn:fval-iteration-inflow}
	\phi_{pf}^{n-1/2,k}
	= \phi_q^{n} + \frac{1}{2}(-\phi_q^{n,k-1} + \nabla\phi_q^{n,k-1}\cdot\mathbf{d}_{qf} +
	\phi_p^{n-1} + \nabla\phi_p^{n-1}\cdot\mathbf{d}_{pf}),
\end{equation}
which is now written, as in the outflow case,
as the upwind cell center value plus a correction term.
To simplify the notation, we denote the correction term as
\begin{equation}\label{eqn:correction-term-inflow}
	\mathcal{D}_{pf}^-\phi^k
	\equiv
	\frac{1}{2}(-\phi_q^{n,k} + \nabla\phi_q^{n,k}\cdot\mathbf{d}_{qf} +
	\phi_p^{n-1} + \nabla\phi_p^{n-1}\cdot\mathbf{d}_{pf}),
\end{equation}
the minus sign indicating that \(a_{pf} < 0\) is negative.

We do similarly at boundary faces. For \(a_{pb} \geq 0\) we have
\begin{equation}
	\begin{split}
		\phi_{pb}^{n-1/2}
		&= \frac{1}{2}
		\left(
			\phi_p^{n} + \nabla\phi_p^{n}\cdot\mathbf{d}_{pb}
			+ \phi_e(\mathbf{x}_{pb}, t^{n-1})
		\right)\\
		&= \phi_p^{n} + \frac{1}{2}
		\left(
			-\phi_p^{n} + \nabla\phi_p^{n}\cdot\mathbf{d}_{pb}
			+ \phi_e(\mathbf{x}_{pb}, t^{n-1})
		\right)\\
		&= \phi_p^{n} + \mathcal{D}^+_{pb}\phi
	\end{split}
\end{equation}
\begin{equation}\label{eqn:fval-bdry-iteration-outflow}
	\phi_{pb}^{n-1/2,k} = \phi_p^{n,k} + \mathcal{D}^+_{pb}\phi^{k-1}
\end{equation}
where
\begin{equation}\label{eqn:correction-bdry-term-outflow}
	\mathcal{D}^+_{pb}\phi^{k} =
	\frac{1}{2}\left(
		-\phi_p^{n,k} + \nabla\phi_p^{n,k}\cdot\mathbf{d}_{pf} + \phi_e(\mathbf{x}_{pb}, t^{n-1})
		\right).
\end{equation}
For \(a_{pb} < 0\) we have
\begin{equation}\label{eqn:fval-bdry-iteration-inflow}
	\begin{split}
		\phi_{pb}^{n-1/2}
		&= \frac{1}{2}
		\left(
			\phi_p^{n-1} + \nabla\phi_p^{n-1}\cdot\mathbf{d}_{pb} + \phi_e(\mathbf{x}_{pb}, t^{n})
		\right)\\
		&= \phi_e(\mathbf{x}_{pb}, t^{n}) + \frac{1}{2}
		\left(
			\phi_p^{n-1} + \nabla\phi_p^{n-1}\cdot\mathbf{d}_{pb} - \phi_e(\mathbf{x}_{pb}, t^{n})
		\right)\\
		&= \phi_e(\mathbf{x}_{pb}, t^{n}) + \mathcal{D}^-_{pb}\phi.
	\end{split}
\end{equation}
Notice that there is no need to compute the inflow boundary term iteratively.

Substituting \eqref{eqn:fval-iteration-inflow}, \eqref{eqn:fval-iteration-outflow},
\eqref{eqn:fval-bdry-iteration-inflow}, \eqref{eqn:fval-bdry-iteration-outflow} to
\eqref{eqn:integral-claw-2ndO-approx-3d} we end up with a linear system
\begin{equation}\label{eqn:system-no-limiter}
	\begin{split}
		\frac{|\Omega_p|}{\Delta t}(\phi^{n,k}_p - \phi_p^{n-1})
		&+ \sum_{f \in \mathcal{F}_p^-}
		\left(
			\phi_q^{n,k} + \mathcal{D}^-_{pf}\phi^{k-1}
		\right) a_{pf}
		+ \sum_{f \in \mathcal{F}_p^+}
		\left(
			\phi_p^{n,k} + \mathcal{D}^+_{pf}\phi^{k-1}
		\right) a_{pf}\\
		&+ \sum_{b \in \mathcal{B}_p^-}
		\left(
			\phi_e(\mathbf{x}_{pb}, t^{n})
			+ \mathcal{D}^-_{pb}\phi
		\right) a_{pb}
		+ \sum_{b \in \mathcal{B}_p^+}
		\left(\phi_p^{n,k} +
		\mathcal{D}^+_{pb}\phi^{k-1}\right) a_{pb} = 0.
	\end{split}
\end{equation}
This yields a system with an M-matrix, which has favorable properties
(\textcolor{red}{reference here}). The inverse of an M-matrix has
positive components. We can think of the solution of the system as
applying the inverse matrix \(\mathbf{M}^{-1}\) to the RHS.
\subsection{Numerical experiments}
\subsection{Limiter}
Solving the system \eqref{eqn:system-no-limiter} for a smooth profile
we get an \(\text{EOC} \approx 2\). For a step function initial profile,
however, in the vicinity of the discontinuity we observe spurious
oscillations. This behavior is inherent for linear high-order schemes in 1D,
proved by Godunov's theorem \cite{1959_Godunov,2009_Toro_BOOK,2007_Hirsch_BOOK,1992_LeVeque_BOOK,1998_Laney_BOOK,2002_LeVeque_BOOK,},
which states that any linear monotone scheme can be at most first-order accurate.
To circumvent this obstacle, we have to employ a nonlinear scheme even for
linear equations, who's coefficients depend on the solution itself. There were many
attempts to do this. This type of limiting process is well established
for 1D problems, see, e.g.~\cite{1977_VanLeer,1984_Sweby,1983_Harten,2004_Kuzmin}. One way of
interpreting limiters is that we limit the high-order correction terms
in a way to ensure boundedness of the numerical solution.
\subsection{Limiter in 1D}
\subsection{Limiter for unstructured mesh}
Thus, for a general mesh, we define the limited face value for outflow \(a_{pf} > 0\)
as follows:
\begin{equation}\label{eqn:faceval-limited-outflow}
	\phi_{pf}^{n-1/2} = \phi_p^n + \Psi_{pf} \mathcal{D}^+_{pf}\phi,
\end{equation}
where \(\Psi_{pf}\) is a limiter chosen in a way to ensure boundedness. In this work,
we bound the limiter
\begin{equation}\label{eqn:limiter-upper-lower-bound}
	0 \leq \Psi_{pf} \leq 1.
\end{equation}
This is equivalent to a convex combination of the
implicit upwind flux \eqref{eqn:faceval-implicit-upwind} and our proposed
high-order flux \eqref{eqn:faceval-no-limiter}.
We can see that, if we write the face value \eqref{eqn:faceval-limited-outflow}
in the form \eqref{eqn:high-order-low-order-combination} as
\begin{equation}
	\begin{split}
		\phi_{pf}^{HO}
		&= \phi_{pf}^{LO} + \Psi_{pf}(\phi_{pf}^{HO} - \phi_{pf}^{LO})\\
		&= \left( 1 - \Psi_{pf} \right)\phi_{pf}^{LO} + \Psi_{pf}\phi_{pf}^{HO}.
	\end{split}
\end{equation}
In the outflow case \(a_{pf} \geq 0\), if \(\Psi_{pf} = 0\), we have the implicit
upwind flux \eqref{eqn:faceval-implicit-upwind}, and in the other extreme case
if \(\Psi_{pf} = 1\) we have the unlimited high-order flux
\eqref{eqn:faceval-no-limiter}. The inflow case \(a_{pf} < 0\) is analogous to the outflow case.

Substituting the limited face value \eqref{eqn:faceval-limited-outflow} to
\eqref{eqn:fvm-discretization-3d}, because \(\Psi_{pf}\) depends on the solution,
yields a nonlinear system of equations. Similarly to the unlimited case,
we solve the equations using the deferred correction procedure.
The face value in the \(k\)-th iteration becomes
\begin{equation}
	\phi_{pf}^{n-1/2,k} = \phi_p^{n,k} + \Psi_{pf}^k \mathcal{D}^+_{pf}\phi^{k-1}.
\end{equation}
We rearrange the terms
\begin{equation}
	\begin{split}
		\frac{|\Omega_p|}{\Delta t}\phi^{n,k}_p
		&+ \sum_{f \in \mathcal{F}_p^-}
		\phi_q^{n,k} a_{pf}
		+ \sum_{f \in \mathcal{F}_p^+}
		\phi_p^{n,k} a_{pf}
		+ \sum_{b \in \mathcal{B}_p^+}
		\phi_p^{n,k} a_{pb}
		=\\
		\frac{|\Omega_p|}{\Delta t}\phi_p^{n-1}
		&-\sum_{f \in \mathcal{F}_p^-}
		\Psi_{pf}^{-, k-1}\mathcal{D}^-_{pf}\phi^{k-1}  a_{pf}
		- \sum_{f \in \mathcal{F}_p^+}
		\Psi_{pf}^{+, k-1}\mathcal{D}^+_{pf}\phi^{k-1}  a_{pf}
		\\
		&-\sum_{b \in \mathcal{B}_p^-}
		\left(
			\phi_e(\mathbf{x}_{pb}, t^{n})
			+ \Psi_{pb}^-\mathcal{D}^-_{pb}\phi
		\right) a_{pb}
		- \sum_{b \in \mathcal{B}_p^+}
		\Psi_{pb}^{+, k-1}\mathcal{D}^+_{pb}\phi^{k-1} a_{pb}.
	\end{split}
\end{equation}
For internal cells, we require for the right-hand side
\begin{equation*}
	\begin{split}
		\phi_{p, vnbd}^{min}
		\leq
		\phi_p^{n-1}
		- \lambda_p \sum_{f \in \mathcal{F}_p^-}
		\Psi_{pf}^{-,k-1}\mathcal{D}^-_{pf}\phi^{k-1} a_{pf}
		- \lambda_p \sum_{f \in \mathcal{F}_p^+}
		\Psi_{pf}^{+,k-1}\mathcal{D}^+_{pf}\phi^{k-1} a_{pf}
		\leq
		\phi_{p, vnbd}^{max}.
	\end{split}
\end{equation*}
Let us define
\begin{equation}
	\phi_{est,p} = \phi_p^{n-1}
	- \lambda_p \sum_{f \in \mathcal{F}_p^-}
	\mathcal{D}^-_{pf}\phi^{k-1} a_{pf}
	- \lambda_p \sum_{f \in \mathcal{F}_p^+}
	\mathcal{D}^+_{pf}\phi^{k-1} a_{pf}.
\end{equation}
First we check in each cell if \(\phi_{est,p} > \phi_{p, vnbd}^{max}\), then we define
\[
	\phi_{target,p} = \phi_{p, vnbd}^{max},
\]
else if \(\phi_{est,p} < \phi_{p, vnbd}^{min}\), then
\[
	\phi_{target,p} = \phi_{p, vnbd}^{min},
\]
otherwise
\[
	\phi_{target,p} = \phi_p^{n-1}
	- \lambda_p \sum_{f \in \mathcal{F}_p^-}
	\mathcal{D}^-_{pf}\phi^{k-1} a_{pf}
	- \lambda_p \sum_{f \in \mathcal{F}_p^+}
	\mathcal{D}^+_{pf}\phi^{k-1} a_{pf}.
\]
We can write this in a compact form as
\begin{equation}
	\phi_{target,p}^n =
	\min
	\left( \phi_{p, vnbd}^{max},
	\max
	\left( \phi_{p, vnbd}^{min}, \phi_{est,p}^n \right) \right).
\end{equation}
We assume, that
\begin{equation}
	\phi_{target,p}^n = \phi_p^{n-1}
	- \lambda_p \sum_{f \in \mathcal{F}_p^-}
	\Psi_{pf}^{-,k-1}\mathcal{D}^-_{pf}\phi^{k-1} a_{pf}
	- \lambda_p \sum_{f \in \mathcal{F}_p^+}
	\Psi_{pf}^{+,k-1}\mathcal{D}^+_{pf}\phi^{k-1} a_{pf},
\end{equation}
\begin{equation}
	\sum_{f \in \mathcal{F}_p^-}
	\Psi_{pf}^{-,k-1}\mathcal{D}^-_{pf}\phi^{k-1} a_{pf}
	+ \sum_{f \in \mathcal{F}_p^+}
	\Psi_{pf}^{+,k-1}\mathcal{D}^+_{pf}\phi^{k-1} a_{pf}
	=
	\frac{\phi_p^{n-1} - \phi_{target,p}^n}{\lambda_p}.
\end{equation}
Furthermore, we want to minimize
\begin{equation}
	\sum_{f \in \mathcal{F}_p^-}
	\left( \Psi_{pf}^{-,k-1} - 1 \right)^2
	\left( \mathcal{D}^-_{pf}\phi^{k-1} a_{pf} \right)^2
	+ \sum_{f \in \mathcal{F}_p^+}
	\left( \Psi_{pf}^{+,k-1} - 1 \right)^2
	\left( \mathcal{D}^+_{pf}\phi^{k-1} a_{pf} \right)^2
\end{equation}
\end{document}