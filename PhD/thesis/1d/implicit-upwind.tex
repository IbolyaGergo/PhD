\documentclass[../thesis.tex]{subfiles}
\begin{document}
\section{Implicit monotone schemes in 1D}
In this section we discuss some stability properties of implicit schemes.

First we show that schemes written in a particular form satisfy the upwind range condition, meaning that the solution is
always bounded by its neighbors in space-time
as it travels along the characteristics.

Next we show that such schemes are also total variation non-increasing (TVNI),
or total variation diminishing (TVD), as it is commonly referred to.

We conclude the section with the observation that if a scheme satisfies the
implicit upwind range condition, then it also satisfies the TVD condition.

Both conditions can be used to construct implicit high-order monotone schemes.
However, as we will see, the upwind range condition is much more straightforward to
apply, only requiring the solution to be bounded between its neighbors. It can be much more cumbersome to require the TVD property of a scheme directly.
\subsection{Upwind range condition for implicit schemes}
Let us consider the implicit upwind scheme in 1 space dimension
\begin{equation}\label{eqn:implicit-upwind-1d}
    \phi_{i}^{n} =
    \phi_{i}^{n-1} - c\left( \phi_{i}^{n} - \phi_{i-1}^{n} \right).
\end{equation}
We assume, that the only unknown value is \(\phi_{i}^{n}\).
To see why that statement holds, notice that the solution of the first order advection
equation requires an inflow boundary condition. Thus, in the first cell, the value \(\phi_{0}^{n}\) is known from the inflow BC, so the only unknown value is \(\phi_{1}^{n}\). Then we compute the next value and so on, until reaching the \(n\)-th cell, where \(\phi_{i-1}^{n}\) is known and the only unknown value is \(\phi_{i}^{n}\).

In this section we want to show that for Courant numbers \(c > 0\), schemes written
in the form of \eqref{eqn:implicit-upwind-1d} satisfy the stability condition
\begin{equation}\label{eqn:implicit-upwind-range-condition}
    \min\left( \phi_{i-1}^{n},\phi_{i}^{n-1} \right)
    \leq
    \phi_{i}^{n}\leq
    \max\left( \phi_{i-1}^{n},\phi_{i}^{n-1} \right),
\end{equation}
which can be interpreted as the upwind range condition\cite{1998_Laney_BOOK}
or data compatibility condition \cite{2009_Toro_BOOK} for implicit schemes.
Notice that min and max is a combination of values from both time steps \(n, n-1\).
This distinguishes the implicit upwind range property from the explicit one.
In the explicit case, the min and max are chosen only from the values at the current time step \(n-1\).
The idea is that the solution travels along characteristics and this value is always bounded by its neighbors in space-time.

To prove \eqref{eqn:implicit-upwind-range-condition}, first we solve \eqref{eqn:implicit-upwind-1d} for the unknown \(\phi_{i}^{n}\).
\begin{equation}\label{eqn:implicit-upwind-solution}
    \phi_{i}^{n} =
    \frac{1}{1+c}\phi_{i}^{n-1} + \frac{c}{1+c}\phi_{i-1}^{n}.
\end{equation}
Notice that
\[
    \frac{1}{1+c} = \frac{1+c-c}{1+c} = 1 - \frac{c}{1+c}.
\]
For convenience, let us denote
\begin{equation}\label{eqn:implicit-upwind-coefficient}
    k = \frac{c}{1+c}.
\end{equation}
Since \(0 \leq c \leq 1+c\), then
\begin{equation}
    \begin{split}
        0 &\leq c \leq 1+c \rightarrow\\
        0 &\leq \frac{c}{1+c} \leq 1 \rightarrow\\
        0 &\leq k \leq 1.
    \end{split}
\end{equation}
Thus, we can rewrite \eqref{eqn:implicit-upwind-solution} as
\begin{equation}\label{eqn:implicit-uwpind-linear-interpolation}
    \phi_{i}^{n} =
    (1-k)\phi_{i}^{n-1} + k\phi_{i-1}^{n},
\end{equation}
where we can see that the solution \(\phi_{i}^{n}\) is a linear interpolation
between \(\phi_{i}^{n-1}\) and \(\phi_{i-1}^{n}\), thus
\eqref{eqn:implicit-upwind-range-condition} holds.
\begin{theorem}
    Let us consider a numerical scheme written in the form
    \begin{equation}\label{eqn:solution-as-interpolation}
    \phi_{i}^{n} =
    (1-k)\phi_{i}^{n-1} + k\phi_{i-1}^{n}.
    \end{equation}
    Assume that \(\phi_{i-1}^{n}\) is known. If the coefficient is bounded \(0 \leq k \leq 1\), then the solution satisfies
    \begin{equation}\label{eqn:implicit-upwind-range-condition-theorem}
        \min\left( \phi_{i-1}^{n},\phi_{i}^{n-1} \right)
        \leq
        \phi_{i}^{n}
        \leq
        \max\left( \phi_{i-1}^{n},\phi_{i}^{n-1} \right),
    \end{equation}
    \begin{proof}
        Without the loss of generality, let us begin with the case where the data satisfies
        \begin{equation}\label{eqn:implicit-upwind-range-condition-theorem-case-1-data}
            \phi_{i-1}^{n} \leq \phi_{i}^{n-1}.
        \end{equation}
        Then we can write
        \begin{equation}
            \begin{split}
                \min\left( \phi_{i-1}^{n},\phi_{i}^{n-1} \right) &= \phi_{i-1}^{n}\\
                \max\left( \phi_{i-1}^{n},\phi_{i}^{n-1} \right) &= \phi_{i}^{n-1}.
            \end{split}
        \end{equation}
        Thus, in the first case we want to show that
        \begin{equation}\label{eqn:implicit-upwind-range-condition-theorem-case-1}
            \phi_{i-1}^{n}
            \leq
            \phi_{i}^{n}
            \leq
            \phi_{i}^{n-1}.
        \end{equation}
        We start by proving that \(\phi_{i}^{n} \leq \phi_{i}^{n-1}\).

        Using \eqref{eqn:implicit-upwind-range-condition-theorem-case-1-data} and
        the fact that \(0 \leq k \leq 1\)
        \begin{equation*}
            \begin{split}
                \phi_{i-1}^{n} &\leq \phi_{i}^{n-1},\\
                k \phi_{i-1}^{n} &\leq k \phi_{i}^{n-1},
            \end{split}
        \end{equation*}
        where \(0 \leq k \leq 1\).
        Then we use \eqref{eqn:solution-as-interpolation} to express \(k \phi_{i-1}^{n}\)
        \begin{equation*}
            \begin{split}
                \phi_{i}^{n} - (1-k)\phi_{i}^{n-1} &\leq k \phi_{i}^{n-1},\\
                \phi_{i}^{n} - \phi_{i}^{n-1} &\leq 0,\\
                \phi_{i}^{n} &\leq  \phi_{i}^{n-1}.
            \end{split}
        \end{equation*}
    It remains to show that \(\phi_{i-1}^{n} \leq \phi_{i}^{n}\).
    \begin{equation*}
        \begin{split}
            \phi_{i-1}^{n} &\leq \phi_{i}^{n-1},\\
            (1-k)\phi_{i-1}^{n} &\leq (1-k) \phi_{i}^{n-1}.
        \end{split}
    \end{equation*}
    Again, using \eqref{eqn:solution-as-interpolation} to express \((1-k) \phi_{i}^{n-1}\)
    \begin{equation*}
        \begin{split}
            (1-k)\phi_{i-1}^{n} &\leq \phi_{i}^{n} - k \phi_{i-1}^{n},\\
            \phi_{i-1}^{n} &\leq \phi_{i}^{n}.
        \end{split}
    \end{equation*}
    Thus,
    \[
        \phi_{i-1}^{n} \leq \phi_{i}^{n} \leq \phi_{i}^{n-1}.
    \]
    The proof for the other case where the data is such that \(\phi_{i-1}^{n} \geq \phi_{i}^{n-1}\) is straightforward.
    \end{proof}
\end{theorem}
\subsection{TVD condition}
The total variation of a mesh function \(\phi^{n} = {\phi_{i}^{n}}\) is defined
as
\begin{equation}
    TV(\phi^{n}) =
    \sum_{i = -\infty}^{\infty}|\phi_{i}^{n} - \phi_{i-1}^{n}|,
\end{equation}
see,~e.g., \cite{1977_VanLeer,1983_Harten,1984_Sweby,1985_Yee,1990_Hirsch_BOOK,1992_LeVeque_BOOK,1996_Jasak,1998_Laney_BOOK,2000_Hoffmann,2002_LeVeque_BOOK,2007_Hirsch_BOOK}.
\begin{theorem}
The solution of a numerical scheme written in the form
\begin{equation}\label{eqn:implicit-upwind-1d-tvd-proof}
    \phi_{i}^{n} =
    \phi_{i}^{n-1} - c\left( \phi_{i}^{n} - \phi_{i-1}^{n} \right),
\end{equation}
for \(c > 0\) and data with compact support,
such that \(\phi_{i}^{n-1} \rightarrow 0 \) as \(i \rightarrow \infty\)
or \(i \rightarrow -\infty\), is total variation non-increasing
\begin{equation}\label{eqn:tvd-condition}
    TV(\phi_{i}^{n-1}) \geq TV(\phi_{i}^{n}).
\end{equation}
\begin{proof}
We start by writing the scheme for cells \(i,i-1\)
    \begin{align}
        \phi_{i}^{n} &=
        \phi_{i}^{n-1} - c\left( \phi_{i}^{n} - \phi_{i-1}^{n} \right),
        \label{eqn:implicit-upwind-i}
        \\
        \phi_{i-1}^{n} &=
        \phi_{i-1}^{n-1} - c\left( \phi_{i-1}^{n} - \phi_{i-2}^{n} \right)
        \label{eqn:implicit-upwind-i-1}.
    \end{align}
Then we subtract \eqref{eqn:implicit-upwind-i-1} from \eqref{eqn:implicit-upwind-i}
to get
\begin{equation}
    \begin{split}
        \phi_{i}^{n} - \phi_{i-1}^{n} =
        \phi_{i}^{n-1} - \phi_{i-1}^{n-1}
        - c\left( \phi_{i}^{n} - \phi_{i-1}^{n}
        - \phi_{i-1}^{n} + \phi_{i-2}^{n} \right),
        \\
        \phi_{i}^{n} - \phi_{i-1}^{n} + c\left( \phi_{i}^{n} - \phi_{i-1}^{n}
        - \phi_{i-1}^{n} + \phi_{i-2}^{n} \right)
        =
        \phi_{i}^{n-1} - \phi_{i-1}^{n-1},
        \\
        (1+c)\left( \phi_{i}^{n} - \phi_{i-1}^{n} \right)
        - c \left( \phi_{i-1}^{n} - \phi_{i-2}^{n} \right)
        = \phi_{i}^{n-1} - \phi_{i-1}^{n-1}.
    \end{split}
\end{equation}
We take the absolute value
\begin{equation}\label{eqn:absolute-value-equivalence}
    \begin{split}
        |(1+c)\left( \phi_{i}^{n} - \phi_{i-1}^{n} \right)
        - c \left( \phi_{i-1}^{n} - \phi_{i-2}^{n} \right)|
        = |\phi_{i}^{n-1} - \phi_{i-1}^{n-1}|.
    \end{split}
\end{equation}
Then we use the triangle inequality \(|a - b| \geq |a| - |b|\) and \(|c~a| = |c||a|\)
\begin{equation}
    |(1+c)\left( \phi_{i}^{n} - \phi_{i-1}^{n} \right)
    - c \left( \phi_{i-1}^{n} - \phi_{i-2}^{n} \right)|
    \geq
    (1+c)|\phi_{i}^{n} - \phi_{i-1}^{n}|
        - c |\phi_{i-1}^{n} - \phi_{i-2}^{n}|,
\end{equation}
Thus, using \eqref{eqn:absolute-value-equivalence}
\begin{equation}
    |\phi_{i}^{n-1} - \phi_{i-1}^{n-1}|
    \geq
    (1+c)|\phi_{i}^{n} - \phi_{i-1}^{n}|
    - c |\phi_{i-1}^{n} - \phi_{i-2}^{n}|.
\end{equation}
To complete the proof, we take the sum of both sides
\begin{equation}\label{eqn:tvd-proof-sums}
    \begin{split}
        \sum_{i=-\infty}^{\infty}|\phi_{i}^{n-1} - \phi_{i-1}^{n-1}|
        &\geq
        \sum_{i=-\infty}^{\infty}(1+c)|\phi_{i}^{n} - \phi_{i-1}^{n}|
        - \sum_{i=-\infty}^{\infty} c |\phi_{i-1}^{n} - \phi_{i-2}^{n}|,
        \\
        \sum_{i=-\infty}^{\infty}|\phi_{i}^{n-1} - \phi_{i-1}^{n-1}|
        &\geq
        (1+c)\sum_{i=-\infty}^{\infty}|\phi_{i}^{n} - \phi_{i-1}^{n}|
        - c \sum_{i=-\infty}^{\infty} |\phi_{i-1}^{n} - \phi_{i-2}^{n}|.
    \end{split}
\end{equation}
To complete the proof, we use the fact that the data has compact support and the solution \(\phi^{n}\) satisfies the upwind range condition
\eqref{eqn:implicit-upwind-range-condition}
\begin{equation}
    \begin{split}
        \min\left( \phi_{i-1}^{n},\phi_{i}^{n-1} \right)
        \leq
        \phi_{i}^{n}\leq
        \max\left( \phi_{i-1}^{n},\phi_{i}^{n-1} \right)
        \text{ for all } i \in (-\infty,\infty),
        \\
        \phi_{-\infty,\infty}^{n-1} = 0,
        \rightarrow
        \phi_{-\infty,\infty}^{n} = 0
        \rightarrow
        \sum_{i=-\infty}^{\infty} |\phi_{i-1}^{n} - \phi_{i-2}^{n}|
        = \sum_{i=-\infty}^{\infty}|\phi_{i}^{n} - \phi_{i-1}^{n}|
        =TV(\phi^{n}).
    \end{split}
\end{equation}
Thus, using \eqref{eqn:tvd-proof-sums} we conclude
\begin{equation}
    \begin{split}
        TV(\phi^{n-1})
        &\geq
        (1+c)TV(\phi^{n}) - c~TV(\phi^{n}),
        \\
        TV(\phi^{n-1})
        &\geq
        TV(\phi^{n}).
    \end{split}
\end{equation}
\end{proof}
\end{theorem}
\subsection{Conclusion}
Since we can transform any scheme written in the form of \eqref{eqn:implicit-upwind-1d}
to a scheme written as \eqref{eqn:implicit-uwpind-linear-interpolation} and
vice versa, then it follows that if a scheme satisfies the implicit upwind range property \eqref{eqn:implicit-upwind-range-condition} it is also TVD.
\subsection{Interpreting implicit schemes as a reconstruction in the current time step}

Explicit finite volume schemes can be interpreted as a reconstruct-evolve-average (REA)
algorithm \cite{1992_LeVeque_BOOK,2002_LeVeque_BOOK,1977_VanLeer}:
\begin{enumerate}
    \item Reconstruct a piecewise polynomial function \(\tilde{\phi}_{i}^{n}(x,t_n)\) for all \(x \in \mathcal{C}_i\) from the known cell averages
    \item Evolve the function to obtain \(\tilde{\phi}_{i}^{n}(x,t_{n+1})\)
    \item Compute the average of this evolved function over each cell to obtain the new cell averages\[\phi_{i}^{n+1} =\frac{1}{\Delta x}\int_{p_i}\tilde{\phi}_{i}^{n}(x,t_{n+1}) \dd{x}.\]
\end{enumerate}
Meaning, we reconstruct the values in a given cell using the cell averages.
For example, the explicit upwind scheme can be interpreted as a piecewise constant reconstruction in the current time step. Then, we evolve this reconstruction and compute
the cell averages in the new time step. We repeat this process until we reach the
desired time.

This interpretation can help construct high-order schemes, where instead of constant
reconstruction we use linear, parabolic, etc., depending on the desired accuracy.
Also, if we make sure that our reconstruction does not generate new extrema, then evolving and averaging operations will neither, meaning that our numerical solution
will be oscillation free.

In this section we would like to show that implicit schemes can also be interpreted similarly by considering that the values travel along characteristics. Thus, the
values in the new time step can be interpreted as values in the current time step.

In this section we restrict our discussion to the simplest problem of advection with positive constant speed \(v > 0\).

We start with the most straightforward piecewise constant reconstruction:
\begin{enumerate}
    \item Reconstruct a piecewise constant function \(\tilde{\phi}_{i}^{n}(x,t_n)\) for all \(x \in \mathcal{C}_i\) from the known cell averages \(\phi_{i}^{n}\).
    Thus,
    \[
        \tilde{\phi}_{i}^{n}(x,t_n) = \phi_{i}^{n} \text{ for all } x \in \mathcal{C}_i.
    \]
    \item Evolve the function using the exact solution of the advection equation
    \[
        \tilde{\phi}_{i}^{n}(x,t_{n+1}) = \tilde{\phi}_{i}^{n}(x - v\Delta t,t_{n})
    \]
    \item Compute the average of this evolved function over each cell to obtain the new cell averages
    \[
        \phi_{i}^{n+1} =\frac{1}{\Delta x}\int_{\mathcal{C}_i}\tilde{\phi}_{i}^{n}(x,t_{n+1}) \dd{x},
    \]
    or, equivalently
    \[
        \phi_{i}^{n+1} =\frac{1}{\Delta x}\int_{\mathcal{C}_i-v\Delta t}\tilde{\phi}_{i}^{n}(x,t_{n}) \dd{x}.
    \]
\end{enumerate}
If the advection with a speed \(v\) over a time step \(\Delta t\) is smaller than a cell width \(\Delta x\), thus the time step is restricted by
\begin{equation}\label{eqn:cfl-1}
    \begin{split}
        &0 \leq v\Delta t \leq \Delta x
        \\
        &0 \leq \frac{v\Delta t}{\Delta x} \leq 1,
        \\
        &0 \leq c \leq 1,
    \end{split}
\end{equation}
where \(c\) is the Courant number, the algorithm yields the first order explicit upwind scheme. The new cell averages are computed as
\begin{equation}\label{eqn:explicit-upwind-1}
    \begin{split}
        \phi_{i}^{n+1} &=\frac{1}{\Delta x}\left(
            v\Delta t~\phi_{i-1}^{n} + \left( \Delta x - v\Delta t \right)\phi_{i}^{n}
        \right)
        \\
        &=\frac{v\Delta t}{\Delta x}\phi_{i-1}^{n}
        + \left( 1 - \frac{v\Delta t}{\Delta x} \right)\phi_{i}^{n}
        \\
        &= c~\phi_{i-1}^{n}
        + \left( 1 - c \right)\phi_{i}^{n}.
    \end{split}
\end{equation}
We can see, that if the CFL condition \eqref{eqn:cfl-1} holds, the new cell
average is a linear interpolation between \(\phi_{i}^{n-1}\) and \(\phi_{i}^{n}\).

Let us compare this with the implicit upwind scheme. We assume that in a cell the
values \(\phi_{i}^{n}\) and \(\phi_{i-1}^{n+1}\) are known, and we want to compute
the new cell average \(\phi_{i}^{n+1}\).

Assume constant reconstruction in the cell \(\mathcal{C}_i\) at the current time step \(t_n\)
as before
\[
    \tilde{\phi}_{i}^{n}(x,t_n) = \phi_{i}^{n} \text{ for } x \in \mathcal{C}_i,
\]
and a constant reconstruction in the cell \(\mathcal{C}_{i-1}\) at the new time step \(t_{n+1}\)
as
\[
    \tilde{\phi}_{i-1}^{n+1}(x,t_{n+1}) = \phi_{i-1}^{n+1}
    \text{ for } x \in \mathcal{C}_{i-1}.
\]
Using the exact solution of the advection equation, we can project the reconstruction
at time \(t_{n+1}\) to the current time \(t_{n}\), by evolving it back in time as
\[
    \tilde{\phi}_{i-1}^{n+1}(x,t_{n})
    = \tilde{\phi}_{i-1}^{n+1}(x + v\Delta t,t_{n+1})
    \text{ for } x \in \mathcal{C}_{i-1} - v\Delta t.
\]
In the explicit case, we computed the new average using the values \(\phi_{i-1}^{n}\),
\(\phi_{i}^{n}\). But now, instead of \(\phi_{i-1}^{n}\) we have \(\phi_{i-1}^{n+1}\).
What we can make is to compute an estimated value \(\phi_{i-1}^{n,*}\) to compute
the new cell average as in \eqref{eqn:explicit-upwind-1}
\begin{equation}
    \phi_{i}^{n+1}
    = c~\phi_{i-1}^{n,*}
    + \left( 1 - c \right)\phi_{i}^{n}.
\end{equation}
We can express the cell average \(\phi_{i-1}^{n,*}\) using \(\phi_{i-1}^{n+1}, \phi_{i}^{n+1}\) by averaging in cell \(\mathcal{C}_{i-1}\)
\begin{equation}
    \phi_{i-1}^{n,*}
    = \left( 1-c \right)\phi_{i-1}^{n+1}
    + c~\phi_{i}^{n+1}.
\end{equation}
Thus, we can solve the 2 equations above for 2 unknowns \(\phi_{i-1}^{n,*}, \phi_{i}^{n+1}\). This yields
\begin{equation*}
    \begin{split}
        \phi_{i-1}^{n,*}
        = \left( 1 - \frac{c}{1+c} \right)\phi_{i-1}^{n+1}
        + \frac{c}{1+c}\phi_{i}^{n},
        \\
        \phi_{i}^{n+1}
        = \frac{c}{1+c}\phi_{i-1}^{n+1}
        + \left( 1 - \frac{c}{1+c} \right)\phi_{i}^{n},
    \end{split}
\end{equation*}
yielding the same result for the new cell average as solving the equation directly.
Also notice that both values are linear interpolations between the known values
\(\phi_{i-1}^{n+1}, \phi_{i}^{n}\).
\end{document}