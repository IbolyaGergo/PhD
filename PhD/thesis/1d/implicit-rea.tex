\documentclass[../thesis.tex]{subfiles}
\begin{document}
\subsection{Interpreting implicit schemes as a reconstruction in the current time step}

Explicit finite volume schemes can be interpreted as a reconstruct-evolve-average(REA)
algorithm \cite{1992_LeVeque_BOOK,2002_LeVeque_BOOK,1977_VanLeer}:
\begin{enumerate}
    \item Reconstruct a piecewise polynomial function \(\tilde{\phi}_{i}^{n}(x,t_n)\) for all \(x \in \mathcal{C}_i\) from the known cell averages
    \item Evolve the function to obtain \(\tilde{\phi}_{i}^{n}(x,t_{n+1})\)
    \item Compute the average of this evolved function over each cell to obtain the new cell averages\[\phi_{i}^{n+1} =\frac{1}{\Delta x}\int_{p_i}\tilde{\phi}_{i}^{n}(x,t_{n+1}) \dd{x}.\]
\end{enumerate}
Meaning, we reconstruct the values in a given cell using the cell averages.
For example, the explicit upwind scheme can be interpreted as a piecewise constant reconstruction in the current time step. Then, we evolve this reconstruction and compute
the cell averages in the new time step. We repeat this process until we reach the
desired time.

This interpretation can help construct high-order schemes, where instead of constant
reconstruction we use linear, parabolic, etc., depending on the desired accuracy.
Also, if we make sure that our reconstruction does not generate new extrema, then evolving and averaging operations will neither, meaning that our numerical solution
will be oscillation free.

In this section we would like to show that implicit schemes can also be interpreted similarly by considering that the values travel along characteristics. Thus, the
values in the new time step can be interpreted as values in the current time step.

In this section we restrict our discussion to the simplest problem of advection with positive constant speed \(v > 0\).

We start with the most straightforward piecewise constant reconstruction:
\begin{enumerate}
    \item Reconstruct a piecewise constant function \(\tilde{\phi}_{i}^{n}(x,t_n)\) for all \(x \in \mathcal{C}_i\) from the known cell averages \(\phi_{i}^{n}\).
    Thus,
    \[
        \tilde{\phi}_{i}^{n}(x,t_n) = \phi_{i}^{n} \text{ for all } x \in \mathcal{C}_i.
    \]
    \item Evolve the function using the exact solution of the advection equation
    \[
        \tilde{\phi}_{i}^{n}(x,t_{n+1}) = \tilde{\phi}_{i}^{n}(x - v\Delta t,t_{n})
    \]
    \item Compute the average of this evolved function over each cell to obtain the new cell averages
    \[
        \phi_{i}^{n+1} =\frac{1}{\Delta x}\int_{\mathcal{C}_i}\tilde{\phi}_{i}^{n}(x,t_{n+1}) \dd{x},
    \]
    or, equivalently
    \[
        \phi_{i}^{n+1} =\frac{1}{\Delta x}\int_{\mathcal{C}_i-v\Delta t}\tilde{\phi}_{i}^{n}(x,t_{n}) \dd{x}.
    \]
\end{enumerate}
If the advection with a speed \(v\) over a time step \(\Delta t\) is smaller than a cell width \(\Delta x\), thus the time step is restricted by
\begin{equation}\label{eqn:cfl-1}
    \begin{split}
        &0 \leq v\Delta t \leq \Delta x
        \\
        &0 \leq \frac{v\Delta t}{\Delta x} \leq 1,
        \\
        &0 \leq c \leq 1,
    \end{split}
\end{equation}
where \(c\) is the Courant number, the algorithm yields the first order explicit upwind scheme. The new cell averages are computed as
\begin{equation}\label{eqn:explicit-upwind-1}
    \begin{split}
        \phi_{i}^{n+1} &=\frac{1}{\Delta x}\left(
            v\Delta t~\phi_{i-1}^{n} + \left( \Delta x - v\Delta t \right)\phi_{i}^{n}
        \right)
        \\
        &=\frac{v\Delta t}{\Delta x}\phi_{i-1}^{n}
        + \left( 1 - \frac{v\Delta t}{\Delta x} \right)\phi_{i}^{n}
        \\
        &= c~\phi_{i-1}^{n}
        + \left( 1 - c \right)\phi_{i}^{n}.
    \end{split}
\end{equation}
We can see, that if the CFL condition \eqref{eqn:cfl-1} holds, the new cell
average is a linear interpolation between \(\phi_{i}^{n-1}\) and \(\phi_{i}^{n}\).

Let us compare this with the implicit upwind scheme
\begin{equation}\label{eqn:rea-implicit-upwind-1d}
    \phi_{i}^{n+1} =
    \phi_{i}^{n} - c\left( \phi_{i}^{n+1} - \phi_{i-1}^{n+1} \right).
\end{equation}
We assume that in a cell the
values \(\phi_{i}^{n}\) and \(\phi_{i-1}^{n+1}\) are known, and we want to compute
the new cell average \(\phi_{i}^{n+1}\).

Assume constant reconstruction in the cell \(\mathcal{C}_i\) at the current time step \(t_n\)
as before
\[
    \tilde{\phi}_{i}^{n}(x,t_n) = \phi_{i}^{n} \text{ for } x \in \mathcal{C}_i,
\]
and a constant reconstruction in the cell \(\mathcal{C}_{i-1}\) at the new time step \(t_{n+1}\)
as
\[
    \tilde{\phi}_{i-1}^{n+1}(x,t_{n+1}) = \phi_{i-1}^{n+1}
    \text{ for } x \in \mathcal{C}_{i-1}.
\]
Using the exact solution of the advection equation, we can project the reconstruction
at time \(t_{n+1}\) to the current time \(t_{n}\), by evolving it back in time as
\[
    \tilde{\phi}_{i-1}^{n+1}(x,t_{n})
    = \tilde{\phi}_{i-1}^{n+1}(x + v\Delta t,t_{n+1})
    \text{ for } x \in \mathcal{C}_{i-1} - v\Delta t.
\]
In the explicit case, we computed the new average using the values \(\phi_{i-1}^{n}\),
\(\phi_{i}^{n}\). But now, instead of \(\phi_{i-1}^{n}\) we have \(\phi_{i-1}^{n+1}\).
What we can make is to compute an estimated value \(\phi_{i-1}^{n,*}\) to compute
the new cell average as in \eqref{eqn:explicit-upwind-1}
\begin{equation}
    \phi_{i}^{n+1}
    = c~\phi_{i-1}^{n,*}
    + \left( 1 - c \right)\phi_{i}^{n}.
\end{equation}
We can express the cell average \(\phi_{i-1}^{n,*}\) using \(\phi_{i-1}^{n+1}, \phi_{i}^{n+1}\) by averaging in cell \(\mathcal{C}_{i-1}\)
\begin{equation}
    \phi_{i-1}^{n,*}
    = \left( 1-c \right)\phi_{i-1}^{n+1}
    + c~\phi_{i}^{n+1}.
\end{equation}
Thus, we can solve the 2 equations above for 2 unknowns \(\phi_{i-1}^{n,*}, \phi_{i}^{n+1}\). This yields
\begin{equation*}
    \begin{split}
        \phi_{i-1}^{n,*}
        = \left( 1 - \frac{c}{1+c} \right)\phi_{i-1}^{n+1}
        + \frac{c}{1+c}\phi_{i}^{n},
        \\
        \phi_{i}^{n+1}
        = \frac{c}{1+c}\phi_{i-1}^{n+1}
        + \left( 1 - \frac{c}{1+c} \right)\phi_{i}^{n},
    \end{split}
\end{equation*}
yielding the same result for the new cell average as solving the equation directly.
This shows that the REA interpretation of the scheme \eqref{eqn:implicit-upwind-1d} is correct.
Also notice that both values are linear interpolations between the known values
\(\phi_{i-1}^{n+1}, \phi_{i}^{n}\).
\end{document}